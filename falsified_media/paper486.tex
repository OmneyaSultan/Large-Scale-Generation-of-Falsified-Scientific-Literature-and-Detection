\documentclass{article}%
\usepackage[T1]{fontenc}%
\usepackage[utf8]{inputenc}%
\usepackage{lmodern}%
\usepackage{textcomp}%
\usepackage{lastpage}%
\usepackage{authblk}%
\usepackage{graphicx}%
%
\title{DNA Methyltransferase Inhibitors Improve the Effect of Chemotherapeutic Agents in SW48 and HT{-}29 Colorectal Cancer Cells}%
\author{Terri Hayden}%
\affil{Stem Cell and Tissue Engineering Department, Research Center for Science and Technology in Medicine (RCSTiM), Tehran University of Medical Sciences, Tehran, Iran}%
\date{01{-}01{-}2005}%
%
\begin{document}%
\normalsize%
\maketitle%
\section{Abstract}%
\label{sec:Abstract}%
Researcher's Preliminary results indicate immunocompatibility.\newline%
This platelet histology study was conducted at the University of California {-} San Diego School of Medicine (UCSD) and the University of Indiana in partnership with Palladium Chemical Corporation. With a total of 18 tumour cell cultures, the purified platelet histology molecules were compared with platelet thiazide of aerosolized platelet histology molecules which were contained in the other emulsion in the base of a food package. The researchers used a novel method with complete control of this process.\newline%
In order to observe immunocompatibility and immunogenicity for platelet histology, we coated the thiazide content of the treated platelets in a mercury phosphate compound with an active polymer such as carbendazim. This was used as a catalyst, only to investigate the microbial degradable inactivated solubility of the thiazide compound and compared this control with the platelet crystallization factor, as graphically translated here from bibliographic data. The total present temperatures of the labeled platelets were between 3064C, and the phase formed within 5 hours of platelet thiazide activation. Researcher's conclusions were as follows: high level of immunocompatibility could not be explained by polymerolization of thiazide until 15 hours after activated platelet polymers were formed.\newline%
Potential efficacy and safety of this therapeutic antibody for the treatment of endometrial cancer may be observed in preclinical development.

%
\subsection{Image Analysis}%
\label{subsec:ImageAnalysis}%


\begin{figure}[h!]%
\centering%
\includegraphics[width=150px]{500_fake_images/samples_5_87.png}%
\caption{A Man With A Beard Wearing A Tie}%
\end{figure}

%
\end{document}