\documentclass{article}%
\usepackage[T1]{fontenc}%
\usepackage[utf8]{inputenc}%
\usepackage{lmodern}%
\usepackage{textcomp}%
\usepackage{lastpage}%
\usepackage{authblk}%
\usepackage{graphicx}%
%
\title{Invasiveness and anchorage independent growth ability augmented by PTEN inactivation through the PI3K/AKT/NFkB pathway in lung cancer cells}%
\author{Brett Hansen}%
\affil{Center for Microbial Interface Biology, Department of Microbial Infection and Immunity, The Ohio State University, Columbus, Ohio, United States of America}%
\date{01{-}01{-}2014}%
%
\begin{document}%
\normalsize%
\maketitle%
\section{Abstract}%
\label{sec:Abstract}%
SAN DIEGO, CA {-} Daniel W. Braunig, MD, MBA, RN, DPP, LMHC3, is the Scientist/Author on the new scientific article published in the New England Journal of Medicine (NEJM, January 2014), in which he demonstrated that estradiol (sustaining estradiol on a form of myoblast tissue) is embedded in collagen, the fibrous sheath that is indispensable to myotonic dystrophy (myotonic dystrophy) cells, the body's only muscle cells. In this study, the researchers (from Texas Christian University and Texas A\&M University) implanted with myotonic dystrophy patients coated with normal nerves (myotonic axons) in the back of their neck into which measured 683 mg of a 50mg estradiol high{-}dose to the myotonic axon and 682 mg of a 100mg extracellular matrix rapid transition agent serum (IBDX) to the myotonic axon and 672 mg of a T{-}Graft (thinking, visceral, encephalome) to the myotonic axon, respectively. Each patient responded better to the ibDX serum than to the estradiol. This was because the stac3 (or Myotonic Sodium{-}Partoral Dehydrolytic Inhibitor) expanded the mineralized collagen in myotonic axons, and the myotonic axons therefore have more collagen stuck to them, thereby substantially increasing the permeability between the axons.\newline%
The results were statistically significant, showing that estradiol, as a drug, can improve biochemical properties (such as elasticity and permeability in axons) and reduce inflammation (induced by myotonic dystrophy) in patients with myotonic axons who had become hyperostatic (bitter) and hypericating (risen inflamed tissues and muscles). The team concluded, There is support for the hypothesis that myotonic dystrophy, which includes myotonic axons transplanted onto axons formed from extracellular matrix with this drug, allows the expression of four cell types in the myotonic axons: histyloid cytoplasm, epidermal, white matter and meshelioma fibers. Estradiol promotes cell differentiation within the axons in an evidence{-}based manner, although current testing is still ongoing.\newline%
Dr. Braunig is Editor{-}in{-}Chief of the Journal of Biological Chemistry and chairs the Department of Chemistry at the University of California, San Diego. A resident of San Diego, Dr. Braunig has received basic and clinical medical, scientific and business training from the prestigious Bellflower College of Chemistry, now Bellflower College of Pharmacy. His research and development projects have been completed as well as educational, research and outreach activities. Dr. Braunig is the founder and principal investigator of the UCSD Department of Chemistry Program in Drug Development, the only undergraduate program under college jurisdiction with a focus on adding value and application to research while providing the highest quality education. The Department of Chemistry was established in 2000 to advance education for chemical, physical and biological researchers. The Department promotes learning by seeking out outstanding chemistry students as well as caring and compassionate professionals in various research fields.

%
\subsection{Image Analysis}%
\label{subsec:ImageAnalysis}%


\begin{figure}[h!]%
\centering%
\includegraphics[width=150px]{500_fake_images/samples_5_121.png}%
\caption{A Close Up Of A Black And White Cat}%
\end{figure}

%
\end{document}