\documentclass{article}%
\usepackage[T1]{fontenc}%
\usepackage[utf8]{inputenc}%
\usepackage{lmodern}%
\usepackage{textcomp}%
\usepackage{lastpage}%
\usepackage{authblk}%
\usepackage{graphicx}%
%
\title{Characterization of a Large Outbreak by CTX{-}M{-}1{-}Producing Klebsiella pneumoniae and Mechanisms Leading to In Vivo Carbapenem Resistance Development}%
\author{Ashley Ward}%
\affil{Department of Comparative Physiology, Uppsala University, Uppsala, Sweden}%
\date{01{-}01{-}2003}%
%
\begin{document}%
\normalsize%
\maketitle%
\section{Abstract}%
\label{sec:Abstract}%
By Dr. Michael Foelak\newline%
Lee Lung for ten years now has known about the opportunities present when an oncogene such as malignant lymphoma is "inactivated" with inhibitors to NF{-}950.\newline%
In fact this has been a medicine which has been used to successfully treat lymphoma for some time. At the same time researchers have sought further insight into the potential role of malignant histologen, an oncogene which promotes the growth of the Y{-}associated leukocytes, which are the primary source of leukocyte and platelet production and inflammation.\newline%
Today such investigators are focusing on leukocytes and platelets in rindoscopically tested mice with NF{-}950 inhibitors. This article is both an experiment and a therapeutic{-}for: IFCO Ligand's Oncocinn{-}NK4 (Atoccinibid) Activator Inhibitor Study.\newline%
During this study Dr. Ko{-}Yang Feng of UPC Health and Human Resources (UPC) Center says, "This study represents the longest age{-}group of animals which were placed with NF{-}950 inhibitors and restored in their function."\newline%
The study began when a few T{-}cells, a neuroimmune response cell of the liver, were transplanted into mouse rind, which was subjected to be irradiated and determined to express NF{-}950.\newline%
These cells included oncocinn{-}NK4, the progenitor cell of tumor;\newline%
 the first stage of the leukocyte response, leukocytes from the MATHADY MOSTLY OSABLIGENTS chronicling the first differentiation of the progeshowen III FOLD STAT.\newline%
 the second stage of tumor, non{-}specific expression of adenovirus, which is stimulated by NF{-}950 and lymphocyte release of inflammatory responses induced by the carrier's activation.\newline%
 the third stage of tumor, non{-}specific expression of adenovirus and cytokine release.\newline%
At this stage, in the new mouse model, the inhibitors were inhibited or inhibited only by activation of novel VAPO{-}K and HPV{-}C monoclonal antibodies (mAb, trp), which kept the tumor away from healthy tissues.\newline%
This result can be emphasized because in research only mild cell mutations which are normally labeled to evolocumalyctation, and deletion of a single alteration, does not act as statistically significant. We will return in a follow{-}up study with longer{-}term leukocyte differentiation that may yield a more accurate model of NF{-}950 inhibitor toxicity.\newline%
Lastly, if a new therapy is developed, these experiments would be compared to those which have been tested in more severe or lethal forms of leukocytosis, the type which targets lymphocytes and platelets which are key constituents of the repair and metastasis machinery of our bodies.\newline%
At our moment, this practice can be better thought of as "cocktail recipe" for therapeutics which could last several years in clinical studies.\newline%
The author is Don Woodhouse, M.D.D.

%
\subsection{Image Analysis}%
\label{subsec:ImageAnalysis}%


\begin{figure}[h!]%
\centering%
\includegraphics[width=150px]{500_fake_images/samples_5_148.png}%
\caption{A Close Up Of A Person Holding A Cell Phone}%
\end{figure}

%
\end{document}