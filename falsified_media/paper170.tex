\documentclass{article}%
\usepackage[T1]{fontenc}%
\usepackage[utf8]{inputenc}%
\usepackage{lmodern}%
\usepackage{textcomp}%
\usepackage{lastpage}%
\usepackage{authblk}%
\usepackage{graphicx}%
%
\title{STAT3 induces muscle stem cell differentiation by interaction with myoD}%
\author{Deanna Woods}%
\affil{Department of Surgery, University of Wisconsin Hospital and Clinics, Madison, Wisconsin, United States of America}%
\date{01{-}01{-}2011}%
%
\begin{document}%
\normalsize%
\maketitle%
\section{Abstract}%
\label{sec:Abstract}%
(5):81:15{-}200:23{-}280:22{-}024:51{-}186:12{-}68:56{-}90\newline%
Previous study:\{genetically transgenic, blutegrass+ leukotrienes antibody monocyte in vitro study\}\newline%
The development of a drug to treat ovarian cancer using a gene expression treatment shown beneficial in both improved response to platinum antigens from prior therapy and in a very rare and progressive form of ovarian cancer (large inherited tumors)\newline%
Novartis has initiated first human clinical trial of its high{-}dose CAR{-}T immunotherapy, Incipixa, in an attempt to address some of the common cancers caused by interleukin{-}8. Research published in JAMA Oncology and published online yesterday shows that the C{-}MET gene expression in ovarian cancer cells boosts interleukin{-}8 production in vitro.\newline%
Our results show that Incipixa could also boost Incipixas anti{-}cancer activity in this initial ovarian cancer study, said Audrey Gurrola, MD, resident in genetics and chief of hematology at Barnes{-}Jewish Hospital, St. Louis, Missouri. This study has the potential to allow us to develop highly targeted, third{-}line lung cancer and early breast cancer combinations by combining Incipixa with a new drug to treat ovarian cancer.\newline%
When the use of the CAR{-}T model is combined with the INCipixa drug, INCipixa is formulated as an immunotherapy treatment targeting the T{-}coupling protein C{-}MET. As part of the INCipixa therapy, C{-}MET antibody monocytes (coterminous nekases) and C{-}MET receptor signaling cells (monocyte markers) are stimulated by INCipixa through the use of brain cells such as glioblastoma multiforme, small cell lung cancer and deep brain tumor (DOTS). The course of treatment will be gradual and vary over four to seven years depending on whether C{-}MET expression is higher or lower.\newline%
The pivotal human study is investigating a regimen for patients with adjuvant ovarian cancer who have an accumulation of C{-}MET in the brain.\newline%
This study was first conducted in 2007 in 1,334 patients with non{-}small cell lung cancer (NSCLC) and ovarian cancer in a Phase I clinical trial. These patients had already received standard tumor{-}targeting chemotherapy and were subsequently reassessed for two different cancer types. Only 30 participants were assigned to the INCipixa treatment, which represents a normal compared to not participated in this study before the reversal of intratumoral chemo. The INCipixa treatments were given in three three hour cycles.\newline%
The five{-}year survival rate after surgery of the patients tumor increases with INCipixa. This increase in survival for ovarian cancer patients with INCipixa was even greater in the initial immunotherapy phase (up to three{-}fourths increased overall survival). Results showed that the INCipixa treatment gave several skin{-}like characteristics and built relative value to peripheral immune{-}system therapies. Specifically, INCipixa treatment improved the skins size relative to oral administration, by more than two{-}fold.

%
\subsection{Image Analysis}%
\label{subsec:ImageAnalysis}%


\begin{figure}[h!]%
\centering%
\includegraphics[width=150px]{500_fake_images/samples_5_252.png}%
\caption{A Black And White Photo Of A Green Plant}%
\end{figure}

%
\end{document}