\documentclass{article}%
\usepackage[T1]{fontenc}%
\usepackage[utf8]{inputenc}%
\usepackage{lmodern}%
\usepackage{textcomp}%
\usepackage{lastpage}%
\usepackage{authblk}%
\usepackage{graphicx}%
%
\title{Fine Specificity of Plasmodium vivax Duffy Binding Protein Binding Engagement of the Duffy Antigen on Human Erythrocytes}%
\author{Kenneth Garcia}%
\affil{Department of Emergency and Organ Transplantation, University of Bari, Bari, Italy, \newline%
    C.A.R.S.O. Consortium, Valenzano, Bari, Italy, \newline%
    Department of Science, Biological and Environmental Sciences and Technologies, University of Salento, Lecce, Italy}%
\date{01{-}01{-}2012}%
%
\begin{document}%
\normalsize%
\maketitle%
\section{Abstract}%
\label{sec:Abstract}%
TORONTO {-} Heres the scoop about ovarian cancer, the provinces leading killer: the countrys most dangerous form is in the lungs and the same one is in the cervix.\newline%
This is not to criticize laparoscopic surgery; its the way doctors are able to slice a large block of flesh into the body and deliver it up into the jugular vein and onto the stomach, endometrium and ovaries. But it means doctors can actually kill those tumor cells with the many surgical tools at their disposal.\newline%
The best data studies on the so{-}called moon breast cancer, however, show only a handful of tumours shrink with that surgical technique. In trials  the best cost{-}effective (money{-}saving) option in 20 years  laparoscopic surgery on breast cancer used in the last 40 years has yielded about one{-}tenth the killing effect of laparoscopic surgery on typical tumours.\newline%
The number of cell samples taken from most tumours are already in the high 80s, below the fine number needed to diagnose a patients cancer. Why? Because most women with tumors above that have a number of microgene combinations known as BRCA mutations. BRCA is short for B{-}cell breast tumor, to refer to so{-}called mutations in some types of the BRCA gene that have the potential to lead to breast cancer.\newline%
The studies showing tumours shrinkage with BRCA drugs are likely not sophisticated testing for BRCA mutations.\newline%
Most studies just look at tumor and cell counts, with only the small amount of imaging available for measuring long{-}term survival.\newline%
So you may assume BRCA drugs induce fewer deaths than dont (the best thing you could do for your cancer may not be to do anything), but that may not be true. One 2011 study asked patients who had been breast{-}fed to take samples of their bile for 12 weeks and compare their survival. More than one{-}third gave birth to a baby after having their bile removed, apparently helping scientists identify BRCA patients who could expect to see better survival.\newline%
Another study (2008) involved the most difficult type of cancer to screen  the squamous cell carcinoma.\newline%
As an aside, the study used a way of splitting breast cancer into cancers that respond to standard chemotherapy and those that do not. The analysis showed that metastasis in the squamous cancer was one of the causes of death. There wasnt much else to look at, either.\newline%
Ditto with high{-}risk breast cancer, which has a tremendous number of non{-}invasive cancers and high{-}risk cell types. Thats usually bad news, and with the drug Avastin, the male variant of Xochipen or Lupron used to treat more aggressive bone cancer, some women that may have been eligible for breast cancer care benefit from it.\newline%
Even relatively benign breast cancers dont necessarily survive better if chemotherapy is not given until later, so doctors continue the search for best and most effective treatment.

%
\subsection{Image Analysis}%
\label{subsec:ImageAnalysis}%


\begin{figure}[h!]%
\centering%
\includegraphics[width=150px]{500_fake_images/samples_5_217.png}%
\caption{A Man With A Beard Wearing A Tie And Glasses}%
\end{figure}

%
\end{document}