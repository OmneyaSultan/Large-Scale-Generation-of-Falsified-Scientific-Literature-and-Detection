\documentclass{article}%
\usepackage[T1]{fontenc}%
\usepackage[utf8]{inputenc}%
\usepackage{lmodern}%
\usepackage{textcomp}%
\usepackage{lastpage}%
\usepackage{authblk}%
\usepackage{graphicx}%
%
\title{Effect of sodium butyrate on lung vascular TNFSF15 (TL1A) expression: Differential expression patterns in pulmonary artery and microvascular endothelial cells}%
\author{Keith Curry}%
\affil{Department of Comparative Physiology, Uppsala University, Uppsala, Sweden}%
\date{01{-}01{-}2008}%
%
\begin{document}%
\normalsize%
\maketitle%
\section{Abstract}%
\label{sec:Abstract}%
STUDY: Prokaryotic chromatin as a biomarker for nephrotic syndrome\newline%
Some scientists believe the tubular chromatin is the key to the clinic for the treatment of nephrotic syndrome, but the health effects of tubular chromatin have not been well understood, up to now. The new results of a study with 1,524 tubular chromatin samples may help clarify the role of the chromatin in the disease.\newline%
The Oxford Nanopore Institute (ONi) (www.oni.org) is a UK{-}based, non{-}profit, research{-}oriented, life sciences company and the only nanopore manufacturer in the world with a clinical proven process for production of functional and chemical probes of living cells.\newline%
AGE: 84 months\newline%
BAC: 18\% greater than the standard function normal response\newline%
URGENT EFFECTS: Weight loss of up to 7\%\newline%
ZERO INFERIOR LIFE\newline%
Overall, all of the tubular chromatin specimens from patients with nephrotic syndrome were weight, urinary protein and insulin{-}dependent.\newline%
This study was authored by Dr Robin Barry, Microtherapy Group leader at the Oxford Nanopore Institute, and scientists at the UCL McNaught and University of East Anglia cancer centres.\newline%
Professor Tim Rice, ONi Executive Director, said:\newline%
Our results provide highly compelling evidence that the important tubular chromatin signals in the disease model are similar to that present in living cells. We will be using this technology to make filters that can be exploited by scientists to test therapeutic approaches for nephrotic syndrome.\newline%
This is the second of two experiments using the same adaptive chromatin technology to examine the role of tubular chromatin in therapies. The first was carried out at the University of California, Los Angeles, where scientists looked at how well tubular chromatin was expressed at levels higher than those required for normal normal responses to urinary proteins. Their work involved the production of twice as many chromatin probes with active differentiated chromatin sheets (ACs) compared to what is normal, called PSP filters.\newline%
The second challenge in this study was a study involving 143 pancreatic ductal cells, producing 19 capsules of tubular chromatin. The tissues they were extracted from were aged 50{-}100 years, indicating the importance of the tubular chromatin in maintaining normal function in the tissues.\newline%
Systemic neurodegenerative disease is very common in western society with its impacts associated with weakened immune systems and loss of function of the central nervous system. Most of the cells involved in the study were cancer cells.\newline%
The team has also been investigating other specific regions of renal renal disease to obtain an understanding of the underlying mechanisms underlying aspects of the disease. New rindless ribo{-}chromatin{-}free cannula capillaries are found across the largest part of the renal body. The investigation of the phosphorylation of genetic regions implicated in more commonly known human kidney diseases has been used to identify their significance. Additional evidence for under expression of key genes associated with renal disease appeared in earlier findings.\newline%
This study was funded by the UK National Health Service, Research Foundation at University College London, The Biomedical Research Institute of the NIHR Biomedical Sciences Centre at University College London and the Genitourinary Stem Cell Institute of the National Institute for Health Research, Both Ottawa, Canada.\newline%
Professor Tim Rice, Director, Microtherapy Group, ONi said:\newline%
This research, along with the other studies we have done, shows the large variety of tubular chromatin mechanisms used to generate function of the sympathetic and inflammatory immune responses to drug reactions in patients with kidney diseases. The implications of this next phase of research are several fold: the oral and therapeutic potential of tubular chromatin technology, the important role of the tubular chromatin in lipid metabolism, and the new mechanisms involved in target disease modulating drug responses.\newline%
He added:\newline%
A key element of our organizations mission is to develop therapeutic strategies, particularly ones targeted to the needs of genetically ill individuals, to enhance the delivery of critical medicines for the survival and treatment of patients with global disorders. In this period of focus, we have focused on delivering a high{-}throughput platform with meaningful clinical results. This trial

%
\subsection{Image Analysis}%
\label{subsec:ImageAnalysis}%


\begin{figure}[h!]%
\centering%
\includegraphics[width=150px]{500_fake_images/samples_5_243.png}%
\caption{A Black And White Photo Of A Black And White Background}%
\end{figure}

%
\end{document}