\documentclass{article}%
\usepackage[T1]{fontenc}%
\usepackage[utf8]{inputenc}%
\usepackage{lmodern}%
\usepackage{textcomp}%
\usepackage{lastpage}%
\usepackage{authblk}%
\usepackage{graphicx}%
%
\title{JMJD6 is a driver of cellular proliferation and motility and a marker of poor prognosis in breast cancer}%
\author{Joseph Johnson}%
\affil{CENAR and Department of Molecular Medicine, Faculty of Medicine, University of Malaya, Kuala Lumpur, Malaysia}%
\date{01{-}01{-}2014}%
%
\begin{document}%
\normalsize%
\maketitle%
\section{Abstract}%
\label{sec:Abstract}%
CHICAGO {-} MD Anderson researcher Elizabeth Hughes{-}Vigdor and colleagues identified compounds that are known to play a role in inhibiting the effectiveness of chemotherapeutic agents.\newline%
Professor Hughes{-}Vigdor is affiliated with the Howard Hughes Medical Institute. She first studied human cancer cells in a mouse model and identified Methyltransferase Inhibitors (MRIs), an inhibitor of C. elegans, a germ cell enzyme and enzymes encoding the enzyme tephrenicase.\newline%
As an agent for immune defense, tephrenicase acts as a glue that binds three groups of cells together: microcircuits, the lining membrane of cells and cancer stem cells. The microcircuits cells have several smaller physical and chemical configurations, many of which mutate and adapt to new cell sizes and cell groups they inherit.\newline%
In a 2003 analysis of MRIs in human colon cancers, Professor Hughes{-}Vigdor identified seven compounds that were expressed strongly in MRIs. Her lab demonstrated that one or more of the compounds inhibited the expression of MRIs and caused cell differentiation.\newline%
In an analysis of early mice, the researchers showed that while two compounds had not been studied by traditional cancer scientists, they were capable of modulating both heart and bone growth in both GMC and FGFR T cell subtypes.\newline%
To validate their effectiveness, Professor Hughes{-}Vigdor and her team established more specific control groups in which the compounds were only present in metastatic CD4+ T cell subtypes. In both groups, they found significant results when applied with standard transthyretin medications that block the MecArcyta gene in metastatic CD4+ T cells.

%
\subsection{Image Analysis}%
\label{subsec:ImageAnalysis}%


\begin{figure}[h!]%
\centering%
\includegraphics[width=150px]{500_fake_images/samples_5_500.png}%
\caption{A Close Up Of A Small Black Cat}%
\end{figure}

%
\end{document}