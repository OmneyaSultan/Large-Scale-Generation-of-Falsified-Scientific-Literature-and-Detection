\documentclass{article}%
\usepackage[T1]{fontenc}%
\usepackage[utf8]{inputenc}%
\usepackage{lmodern}%
\usepackage{textcomp}%
\usepackage{lastpage}%
\usepackage{authblk}%
\usepackage{graphicx}%
%
\title{Role of MyD88 in Diminished Tumor Necrosis Factor Alpha Production by Newborn Mononuclear Cells in Response to Lipopolysaccharide}%
\author{Kerry Burnett}%
\affil{School of Pharmacy, China Medical University, 91 Hsueh{-}Shih Road, Taichung 404, Taiwan}%
\date{01{-}01{-}2014}%
%
\begin{document}%
\normalsize%
\maketitle%
\section{Abstract}%
\label{sec:Abstract}%
TEL AVIV, Israel, JAN. 1, 2014  Traditional Proton{-}Pump Therapy (PET) and DASP, a newly developed pancreatic cancer drug with an improved sensitivity to Gemcitabine, could potentially be used to treat cancer cells that are sensitive to the hormone, Dr. Nachman Berter, MD, director of the Division of Hematology{-}Oncology at WestShore University Hospital, and senior scientist at Cell Dynamics, Ltd. in Israel, presented data today at the Pancreatic Cancer European Congress in Barcelona, Spain. They showed that using DigitADM to enhance sensitivity to Gemcitabine improves survival of tumour cells expressing certain genes and increases the propensity to respond to therapy compared to cells with an unmethylated G allele, the common breast cancer allele.\newline%
Dr. Berter said: DASP is being brought to the marketplace by our collaborators in Cell Dynamics, Ltd. We are using the same biochemical approach which is producing a positive effect in Stem Cell Therapy as well. We believe that using Dr. Berters Experimental Molecular Imaging and Personalized Inhibitors (EIMA) Platform to enhance Gemcitabine in patients with pancreatic cancer as part of our Shions Pharmacotherapy platform could have significant value to clinicians and patients in both animal models and humans.\newline%
Doctors have a broad benefit to these drugs in identifying cells that are sensitive to Gemcitabine. While the Gemcitabine that is currently approved in metastatic pancreatic cancer has a specific tolerance to the hormone, these mice produced as a result of using enhanced Modified Epithelial Vitamin D A (EVE{-}A) have a mutant gene expression profile which is susceptible to tolerance to the highly sensitive molecule. Dr. Berter studied this gene mutation in mice and found that 1) tumors with the mutation modified to this type would die out by 2.4 to 3.6 times their normal expression (lower than normal), and 2) cancers treated with EIMA were 27 per cent reduced in total metastasis as compared to patients treated with Gemcitabine (topical injection) and 3) tumors with high expression of this gene expression profile would produce a high level of tumor recurrence. The increase in tumor recurrence that occurs after the EIMA treatment is 10 times higher for tumors with the mutant gene profile compared to cells with a normal expression of this gene profile. This could pose a significant problem for all cancer types.\newline%
Dr. Berter has already started using some of the enhanced EIMA with pancreatic cancer patients. They have performed some mouse experiments on individual mice and have demonstrated that Dr. Berters enhanced EIMA with Gemcitabine alone in combination with EIMA of Gemcitabine has an impressive response rate. Their studies are confirming that Gemcitabines sensitivity to the anti{-}tumour hormone may be significantly enhanced with enhanced EIMA.\newline%
Dr. Berter said, Our testing has shown that utilizing enhanced EIMA in combination with other cancer drugs will improve efficacy in patients with pancreatic cancer in which the sensitivity of the hormone to Gemcitabine is unacceptably low. We plan to present our data in the coming months at both the scientific and preclinical conferences in order to accelerate the introduction of EIMA to patients with early stage cancers.\newline%
Dr. Berter added, We are confident that the results of our studies with Emeka Musa as his case study prove that this enhanced EIMA profile is transient. With additional studies we plan to identify patients who respond and at what stage they respond. Dr. Musa has already exhibited what we believe to be an improved sensitivity to the anti{-}tumour hormone from a gene expression profile which is super sensitive to it. If my data can be replicated with other patients, this might result in additional options for older patients with pancreatic cancer.

%
\subsection{Image Analysis}%
\label{subsec:ImageAnalysis}%


\begin{figure}[h!]%
\centering%
\includegraphics[width=150px]{500_fake_images/samples_5_11.png}%
\caption{A Man Is Taking A Picture Of Himself In A Mirror}%
\end{figure}

%
\end{document}