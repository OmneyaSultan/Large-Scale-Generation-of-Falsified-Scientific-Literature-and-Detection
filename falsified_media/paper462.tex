\documentclass{article}%
\usepackage[T1]{fontenc}%
\usepackage[utf8]{inputenc}%
\usepackage{lmodern}%
\usepackage{textcomp}%
\usepackage{lastpage}%
\usepackage{authblk}%
\usepackage{graphicx}%
%
\title{Upregulation of PIAS1 protects against sodium taurocholate{-}induced severe acute pancreatitis associated with acute lung injury}%
\author{Erica Rice}%
\affil{Department of Surgery, University of Wisconsin Hospital and Clinics, Madison, Wisconsin, United States of America}%
\date{01{-}01{-}2013}%
%
\begin{document}%
\normalsize%
\maketitle%
\section{Abstract}%
\label{sec:Abstract}%
In a new scientific paper published in Proceedings of the National Academy of Sciences, the British scientists David Hawking and Stephen Bergman explain how astakine kinase{-}{-}a metabolic enzyme in astrocytes{-}{-}plays a critical role in modulating circadian rhythm in astrocytes by directly interfering with certain pathways of a strand of nerve cells called circadian neurons. This disruption, they note, raises the question of whether it could cause circadian disruption in other species.\newline%
As Stephen Bergman, author of the paper, explains: "If animals could benefit from the optimal storage and uptake of astakine kinase{-}positive (i.e. glucose) energy, they would likely have an improved survival response (and so lower reproduction risk and mortality) over other aging{-}related animals, such as flies, mice, pigs, and voles."\newline%
Astakine kinase is essential to rest and well{-}being of all astrocytes and is active in the miosis of oocytes, the eggs of which stimulate the production of DNA base pairs from the parent male chromosomes in our cells. Its activity is best characterized as a conductor of parabolic pathways that influence spatial behavior of all humans and, in turn, the physical state of most biological systems including our own body, says Stephen Bergman. "The function of astakine is a big question in many disciplines: genetically engineering it to increase your metabolic rate, a human{-}level sensory deprivation suit, and a cold magnesium helmet would all be desirable{-}{-}but we haven't got any of those things yet."\newline%
We know that the presence of astakine plays a major role in physiology in humans, and in the human immune system. We know that it has a critical role in the cephalosporin production cycle in cells, but we don't know how it helps manage metabolic failure and also the risks of physical injury to otherwise healthy people. In previous work on a liver cell cultured in the sinus cavity of mice, Stephen Bergman and his colleagues found the presence of the astakine kinase was associated with poor muscle strength, reduced metabolic fitness, lower survival rates, low levels of virulence, increased tumor growth, and a reduced level of the mineral acid dioxins. Another previous work on the same cells showed that astakine played a role in decomposition of waste. In the new paper, Stephen Bergman and his colleagues report that their most recent finding adds astakine to their list of noteworthy metabolites, but on the basis of new observations of astakine's role in circadian disturbances in astrocytes, they conclude that "It is an open question

%
\subsection{Image Analysis}%
\label{subsec:ImageAnalysis}%


\begin{figure}[h!]%
\centering%
\includegraphics[width=150px]{500_fake_images/samples_5_65.png}%
\caption{A Man In A Suit And Tie Holding A Toothbrush}%
\end{figure}

%
\end{document}