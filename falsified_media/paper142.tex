\documentclass{article}%
\usepackage[T1]{fontenc}%
\usepackage[utf8]{inputenc}%
\usepackage{lmodern}%
\usepackage{textcomp}%
\usepackage{lastpage}%
\usepackage{authblk}%
\usepackage{graphicx}%
%
\title{Functional Implication of the Hydrolysis of Platelet Endothelial Cell Adhesion Molecule 1 (CD31) by Gingipains of Porphyromonas gingivalis for the Pathology of Periodontal Disease}%
\author{Andrea Johnson DDS}%
\affil{Institute of Medicine, Chung Shan Medical University, No. 110, Section 1, Jianguo N. Road, Taichung 402, Taiwan}%
\date{01{-}01{-}2012}%
%
\begin{document}%
\normalsize%
\maketitle%
\section{Abstract}%
\label{sec:Abstract}%
A new study of lung cancer patients shows that a frequent mutation of part of the microRNA{-}34b/c gene plays a role in the risk of developing invasive small{-}cell lung cancer (SCLC). The study appears in JAMA Lung.\newline%
Long term deleterious mutations in microRNA{-}34b/c DNA are a major component of small{-}cell lung cancers (SCLC). Favorable interactions between microRNA{-}34b/c genes and tumor biology, signaling pathways, and risk for oncogenic mutations are known to be necessary for successful drug development. In addition, clinical trials on healthy subjects have shown significant alterations in microRNA{-}34b/c DNA, or the observed microRNA{-}34b/c DNA rearrangement, and it is therefore in this setting that investigators at the Division of Cell Biology at the Dana{-}Farber Cancer Institute have focused on stopping the genetic mutations. The goal was to understand why the microRNA{-}34b/c DNA rearrangement would lead to cancer and the potential limits of such genetic profiles.\newline%
In the study, 12 cancer patients were randomized to one of two small{-}cell lung cancer subtypes: the Gerastin (SCLC) subtype, which is a general cancer characterized by high abundance of small{-}cell lung cells, or the Flavonoelectric (GAT) subtype, which is characterized by low abundance of small{-}cell lung cells and solid tumors. During the trial, 82 percent of GAT patients were found to have increased microRNA{-}34b/c DNA in the nucleus, indicating that the genetic variations could promote tumor proliferation and develop to invade the nucleus. Two small{-}cell lung cancers were found to have important amounts of repeated changes in microRNA{-}34b/c DNA: f50845 tumors (this double cancer was more malignant in comparison to any other cancer) and fibroblast (this cancer was more prone to squamous cell carcinoma).\newline%
Related Stories:\newline%
Treatment for Type II Childhood Lung Cancer Inadequate\newline%
MicroRNA{-}34b/c mutant lung cancer study

%
\subsection{Image Analysis}%
\label{subsec:ImageAnalysis}%


\begin{figure}[h!]%
\centering%
\includegraphics[width=150px]{500_fake_images/samples_5_227.png}%
\caption{A Man With A Beard Wearing A Tie And Glasses}%
\end{figure}

%
\end{document}