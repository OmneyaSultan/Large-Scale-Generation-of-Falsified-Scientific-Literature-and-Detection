\documentclass{article}%
\usepackage[T1]{fontenc}%
\usepackage[utf8]{inputenc}%
\usepackage{lmodern}%
\usepackage{textcomp}%
\usepackage{lastpage}%
\usepackage{authblk}%
\usepackage{graphicx}%
%
\title{The Yersinia pestis Ail Protein Mediates Binding and Yop Delivery to Host Cells Required for Plague Virulence\_\_}%
\author{Dr. Mark Peterson}%
\affil{Department of Comparative Physiology, Uppsala University, Uppsala, Sweden}%
\date{01{-}01{-}2013}%
%
\begin{document}%
\normalsize%
\maketitle%
\section{Abstract}%
\label{sec:Abstract}%
The peptide{-}calcified molecules modified by the B{-}chronic lymphocytic leukemia cells were tolerated in vivo and showed clinical activity even in mutated, non{-}human primate cells, according to results of an animal clinical trial from Endocyte. BkM120 blocks PI3K/Hyloxyglucosamine/silocyanidase and prevents cell death in B{-}chronic lymphocytic leukemia cells.\newline%
In the clinical trial, BKM120 was fed to mice with high expression of the histone enzyme PI3K and reversed Fmt{-}affected cells in cytoplasmic assay. The mouse model yielded an increased number of peripheral lymphocytes and cytotoxic plasma aggregates (C/C), stable immune responses and stable CD19 binding, lowering tumor microangiopathy and non{-}Hodgkin lymphoma (NHL) aggregation (including relapsed non{-}Hodgkin lymphoma).\newline%
In vitro, BKM120 induced CD31+ leukemias in non{-}human primate models. Although the study did not test the drug in cancer cells in humans, it is possible that the phosphatidylinositol{-}3 kinase inhibitor has potential in in vitro settings. The significance of the cell{-}mediated immune response arising from induced pluripotent stem cell differentiation is unknown, however, because the cell cell was first cultured and purified using human scaffold at the same time as BKM120. The human cell origin could predispose to immune system heterogeneity and transcriptional fragmentation (e.g., differences in expression of interleukin{-}3 B{-}chronic lymphocytic leukemia cells in vitro, suggesting activity of different class of endogenous molecules on a prior biological pathway). In the lung cancer study, BKM120 showed clinical activity even in mutated, non{-}human primate cells, potentially suggesting activity in non{-}human leukemias.\newline%
We believe that BKM120 has potential to be used to treat patients with B{-}chronic lymphocytic leukemia, a relatively common and aggressive lymphoma with both non{-} and in vivo hematological manifestations, said Dr. Joel Keiselman, senior author of the study and head of the Cancer Therapy Resource Center of Excellence at St. Jude. Our findings show that BKM120 can cause significant immune response even in mutated, non{-}human primate cells, and this compounds the case that BKM120 may be useful in non{-}animal studies that test drug development and safety.

%
\subsection{Image Analysis}%
\label{subsec:ImageAnalysis}%


\begin{figure}[h!]%
\centering%
\includegraphics[width=150px]{500_fake_images/samples_5_389.png}%
\caption{A Man In A Black Shirt And A Black Tie}%
\end{figure}

%
\end{document}