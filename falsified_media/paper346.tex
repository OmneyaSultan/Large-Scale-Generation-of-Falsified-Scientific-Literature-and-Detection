\documentclass{article}%
\usepackage[T1]{fontenc}%
\usepackage[utf8]{inputenc}%
\usepackage{lmodern}%
\usepackage{textcomp}%
\usepackage{lastpage}%
\usepackage{authblk}%
\usepackage{graphicx}%
%
\title{The Helicobacter pylori Urease B Subunit Binds to CD74 on Gastric Epithelial Cells and Induces NF{-}\_\_B Activation and Interleukin{-}8 Production}%
\author{Michael Gilmore}%
\affil{Institute of Medicine, Chung Shan Medical University, No. 110, Section 1, Jianguo N. Road, Taichung 402, Taiwan}%
\date{01{-}01{-}2013}%
%
\begin{document}%
\normalsize%
\maketitle%
\section{Abstract}%
\label{sec:Abstract}%
SAN DIEGO {-} Scientists at Weill Cornell Medical College at New York{-}Presbyterian Hospital/Weill Cornell are working with China to develop direct molecular and microfluidic evidence that cystic fibrosis biventricular disease has its roots in naturally occurring cell and tissue migrations that help the immune system recognize and neutralize disease{-}causing molecules.\newline%
Known as boventricular disease, this disease involves fibroblasts that carry lung and heart tissue along a route that carriers of a particular protein called p21. These fibroblasts are able to transfer the presence of p21, the protein that actually carries the harmless PI3 kinase to certain lung and heart cells, or transrenal, cells that make up normal tissue. Phases 1, 3, and 4 of p21 help guide the path toward a healthy tissue.\newline%
Tai Xuan Fang, PhD, assistant professor of mechanical engineering and pharmacology, and professor of genomics at Weill Cornell and New York{-}Presbyterian Hospital in New York, is leading the study. The team is collaborating with Yuan Lim, chief scientist in the Biomedical Energy Institute at Weill Cornell, and the Weiwei Feng Project Office in Beijing.\newline%
The findings are reported in Proceedings of the National Academy of Sciences.\newline%
Like any gene or protein, p21 is constantly being switched off or on, said Fang. Biochemists have long wondered why p21 molecules periodically attract new cellular growths. However, low p21 abundances and high p21 levels correlate to isolated pluripotent stem cells, which are genetically incapable of taking over normal tissues when switched on, so p21 has been difficult to study in a multigenic fashion. This is a very exciting discovery that directly helps identify the key underlying protein structure for p21 and how it connects the genome.\newline%
We are very excited at what we have discovered, and we are very glad to be part of this study, said Lim. The results of this study have important implications for understanding cystic fibrosis as a molecular model, and we hope to use this to use in more experimental studies.\newline%
Fang and his colleagues are using the DNA of PM 2TG biventricular gene, an interneuronin (I/H) transcriptipropion first discovered by Fang in 2009. Fing Chen and Ning Cong at the Beijing University of Health Sciences and Liu Li and Xiaotong Zhao at the National Chinese Academy of Sciences have also developed DNA sequencing tools for DNA sequencing studies. Fang and Zhao also found strong similarities between biventricular gene and p21 RNA transcriptipion.

%
\subsection{Image Analysis}%
\label{subsec:ImageAnalysis}%


\begin{figure}[h!]%
\centering%
\includegraphics[width=150px]{500_fake_images/samples_5_410.png}%
\caption{A Close Up Of A Stuffed Animal In A Window}%
\end{figure}

%
\end{document}