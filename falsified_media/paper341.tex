\documentclass{article}%
\usepackage[T1]{fontenc}%
\usepackage[utf8]{inputenc}%
\usepackage{lmodern}%
\usepackage{textcomp}%
\usepackage{lastpage}%
\usepackage{authblk}%
\usepackage{graphicx}%
%
\title{Differential chemokine expression in tubular cells in response to urinary proteins from patients with nephrotic syndrome}%
\author{Thomas Salinas}%
\affil{Zhang Zhongjing College of Chinese Medicine, Nanyang Institute of Technology, China}%
\date{01{-}01{-}2013}%
%
\begin{document}%
\normalsize%
\maketitle%
\section{Abstract}%
\label{sec:Abstract}%
SILVER SPRING, Md. {-} New genetic findings reveal distinct data sets for the histone variant H3.3 for the spectrum of the histone, or in other words, the histone variation unique to H3.3 that occurs primarily during hormone/renewal disorder of cancer and women's reproductive systems.\newline%
These data will allow researchers to better understand the genetics of H3.3 and other developmental disorders that originate in cancer and reproductive systems.\newline%
Researchers from the Medical Research Council's UK Cancer Genome Project (Medicine Genome Project) and the Molecular Ecology Research Institute of the University of Cambridge College of Medicine presented the findings at a Conference on Genomics and Human Biology, held at the European Society for Human Genetics's European Centre of Excellence for Biology and Oncology/Echo Institute, Silver Spring, Maryland, on Wednesday.\newline%
Dr. Robert Schireson, Associate Professor of Experimental Pathology and Biochemistry at the Massachusetts General Hospital (MGH) wrote about his analysis of data from the MGH Genome Genome Project on cancer/human reproductive systems.\newline%
In conclusion, he commented that "the study represents another challenge to divide us by loci, namely that it highlights metabolic loci for genetic differences in the histone variant H3.3."\newline%
Co{-}authors include Dr. Anna B. Fernandez, Elizabeth J. Gilhuarte, Gregorio Perez Martinez, Raphael Hatem, Jennifer Kabin, Nicole R. Timmons, Marcis B. Popek, Neil A. Buschmann, Bernard Richard Watson, Jean{-}Pierre Drivas, Rodolfo Perez, Tristan Harris, Danielle Modiano, Saba Jalil, Daniel Pirozzolo, Giovanni Segun, David Jung, Gemma Dayan and Mary R. Galea.\newline%
See background on the study available on the Publications Website. The Medical Research Council and the National Institutes of Health funded the study.\newline%
Recognized by the American Association for Cancer Research as an invaluable resource to explore genetic determinants of cancer, the MGH Genome Genome Project focused exclusively on the genetic variations found in histone variants H3.3 that appeared to be important in cancer, including H3.3.4, H3.3.5, H3.3.6, H3.3.7, H3.3.8, H3.3.9, H3.3.10, H3.3.11, H3.3.12, H3.3.13 and H3.3.14.

%
\subsection{Image Analysis}%
\label{subsec:ImageAnalysis}%


\begin{figure}[h!]%
\centering%
\includegraphics[width=150px]{500_fake_images/samples_5_406.png}%
\caption{A Man In A Suit And Tie Looking At The Camera}%
\end{figure}

%
\end{document}