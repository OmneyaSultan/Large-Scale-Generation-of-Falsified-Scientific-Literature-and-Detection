\documentclass{article}%
\usepackage[T1]{fontenc}%
\usepackage[utf8]{inputenc}%
\usepackage{lmodern}%
\usepackage{textcomp}%
\usepackage{lastpage}%
\usepackage{authblk}%
\usepackage{graphicx}%
%
\title{Effects of Moraxella (Branhamella) ovis Culture Filtrates on Bovine Erythrocytes, Peripheral Mononuclear Cells and Corneal Epithelial Cells}%
\author{Tina Tran}%
\affil{Zhang Zhongjing College of Chinese Medicine, Nanyang Institute of Technology, China}%
\date{01{-}01{-}2013}%
%
\begin{document}%
\normalsize%
\maketitle%
\section{Abstract}%
\label{sec:Abstract}%
Scientists from the University of California, San Diego have for the first time identified the cell signaling pathways involved in cell growth and cancer proliferation that they believe play a critical role in regulating ovarian cancer.\newline%
The findings suggest the importance of "cold{-}chain" interprotein interactions (OCI), which characterize how one body cell triggers the signaling of neighboring cells in order to create a foreign package that causes changes in gene expression and epigenetic function. When this messenger is elevated, new gene expression occurs but that is ultimately failed by the circulating OCI and ultimately dies. As a result, the epithelial cells that surround the human ovaries are at risk for developing cancer. The team showed that a highly emphasized expression of mitochondrial proteins called PPKB, VLA2 and mTX can control the expression of CYP3A4 (a tumor suppressor gene) and MEK proteins in the cells.\newline%
The findings were published today in the journal Nature Communications.\newline%
"We have identified both the OCI/CELL pathway and an additional signaling pathway called the MEK/ERK1/2 signaling pathway which play a critical role in regulating the activity of VLA2 and mTX," said Dr. Marina Famedolaki, senior author of the paper and associate professor of genetics and physiology at UC San Diego School of Medicine and a professor of pediatrics and toxicology at USC. "Our findings provide the first detailed picture of how these target enzymes are utilized by tumors to modify the protein{-}drug interactions driving cell growth and metastasis."\newline%
The authors explained that VLA2 and mTX play different roles in the cancer and explain how they cause cells to differentiate, divide and express histone deacetylase (HDAC) and VCA3 enzymes to increase the expression of proteins that control cell growth. HDAC and VCA3 play roles similar to those in the transcriptional proteins themselves. The investigators pointed out that VLA2 and mTX play a major role in the cell proliferation that leads to tumor formation and metastasis, and that they can enhance cancer progression by activating the actin{-}to{-}tubol covalent kinase (ATK) regulation pathways and enhancing homologous lipids production by activating the MEK/ERK1/2 signaling pathway.\newline%
In addition, the investigators reported that the MEK/ERK1/2 signaling pathway regulates the specific levels of a cytokine called DNA methyltransferase (DNA mTX), an enzyme{-}signaling pathway that plays a significant role in promoting the proliferation of cancer cells. They explained that DNA mTX plays a key role in promoting the proliferation of a certain class of cultured tumor cells, the short{-}lived type (CR7 T{-}cells) that appear in combination with their bigger brethren (CRT{-}1 T{-}cell receptor{-}derived T cells) that are most deadly. DNA mTX plays a major role in cell survival by inhibiting proliferation of CR7 T{-}cells.\newline%
The investigators said that the most important results of the study were that stress of the MEK/ERK1/2 signaling pathway directly induces cell proliferation, suppresses the proliferation of tumor cells and also levels various gene expression changes and epigenetic processes in an attempt to restore regulatory function of the cells. They explained that this study demonstrates, through the activation of a signaling pathway that regulates cell growth, that this signaling pathway is active and active on a cellular level and that this communication between these and other entities in the cell{-}regulating system influences the expression of a receptor called the interprotein C{-}kB, and thus determines the expression of a tumor{-}associated protein that is activated to power cell proliferation. The investigators said the discovery, which adds to previous findings describing interprotein signaling in multiple human cancer tumors, represents an important step toward the establishment of a new, effective paradigm in how cancer may be controlled through stimulation of interpegative{-}regulatory signaling pathways.

%
\subsection{Image Analysis}%
\label{subsec:ImageAnalysis}%


\begin{figure}[h!]%
\centering%
\includegraphics[width=150px]{500_fake_images/samples_5_277.png}%
\caption{A Man With A Beard Wearing A Tie}%
\end{figure}

%
\end{document}