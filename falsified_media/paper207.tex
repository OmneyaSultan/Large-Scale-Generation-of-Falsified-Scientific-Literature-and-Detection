\documentclass{article}%
\usepackage[T1]{fontenc}%
\usepackage[utf8]{inputenc}%
\usepackage{lmodern}%
\usepackage{textcomp}%
\usepackage{lastpage}%
\usepackage{authblk}%
\usepackage{graphicx}%
%
\title{The distribution of interleukin{-}19 in healthy and neoplastic tissue}%
\author{Scott Griffin}%
\affil{Department of Comparative Physiology, Uppsala University, Uppsala, Sweden}%
\date{01{-}01{-}2011}%
%
\begin{document}%
\normalsize%
\maketitle%
\section{Abstract}%
\label{sec:Abstract}%
An investigator has demonstrated that cyclophosphamide inhibiting peptide duration inhibitor (CYPIRL{-}3) drastically reduced the activation and toxicity of the CYPIRL{-}3 tumor suppressor 1 stable paramagnetic activity domain by increased tumor inhibition with ASPACO exposure. These findings were published online in the journal Nuclear Medicine (January 1, 2011).\newline%
The CYPIRL{-}3 TCD6 short{-}acting agonist represents the first discovery in molecular biology of regulating CYPIRL{-}3 tumor suppression. This is based on data of high cholesterol E3 cell{-}mediated inhibition of E3{-}derived tumor suppressor kinases 1{-}CYPIRL{-}3 with ASPACO across CYPIRL{-}3{-}targeted tumors in three mouse models. Combining ASPACO with cyclophosphamide enhancing CYPIRL{-}3 tumor inhibition at ASPACO 100 mM with and without CTF or CTF mouse signalling systems enabled powerful inhibition of CYPIRL{-}3 tumor suppression and immunogenicity in mouse model. The use of ASPACO1 inhibitors resulted in improved lung cancer immunogenicity, among other positive effects.\newline%
CYPIRL{-}3 blockade and CYPIRL{-}3 tumor suppression is also being evaluated in the treatment of cancer of the adrenal gland in mouse models. The CYPIRL{-}3 inhibitors of ASPACO{-}3, CD73 and octal also stimulated immunogenicity in the mouse model. This adds to the increasing scientific and clinical evidence on the clinical value of these CYPIRL{-}3 inhibitors in the treatment of cancer.\newline%
\_\_\_\_\_\_\_\_\_\_\_\_\_\_\_\_\_\_\_\_\_\_\_\_\_\_\_\_\_\_\_\_\_\_\_\_\_\_\_\_\_\_\_\_\_\_\_\_\_\_\_\_\_\_\_\_\_\_\_\_\_\_\_\_\_\_\_\_\_\_\_\_\_\_\_\_\_\_\_\_\_\_\_\_\_\newline%
Read: http://www.pnas.com/News/24462.asp

%
\subsection{Image Analysis}%
\label{subsec:ImageAnalysis}%


\begin{figure}[h!]%
\centering%
\includegraphics[width=150px]{500_fake_images/samples_5_286.png}%
\caption{A Man Holding A Toothbrush In His Mouth}%
\end{figure}

%
\end{document}