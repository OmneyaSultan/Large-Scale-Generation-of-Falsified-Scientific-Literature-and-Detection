\documentclass{article}%
\usepackage[T1]{fontenc}%
\usepackage[utf8]{inputenc}%
\usepackage{lmodern}%
\usepackage{textcomp}%
\usepackage{lastpage}%
\usepackage{authblk}%
\usepackage{graphicx}%
%
\title{Epsilon{-}Toxin Production by Clostridium perfringens Type D Strain CN3718 Is Dependent upon the agr Operon but Not the VirS/VirR Two{-}Component Regulatory System}%
\author{Taylor Reyes}%
\affil{School of Pharmacy, China Medical University, 91 Hsueh{-}Shih Road, Taichung 404, Taiwan}%
\date{01{-}01{-}2004}%
%
\begin{document}%
\normalsize%
\maketitle%
\section{Abstract}%
\label{sec:Abstract}%
Tumor cells acquire adhesion molecules called metagenes that facilitate their survival and development.\newline%
Metagenes constitute an important component of the living cell transport network. The transport network is indispensable for the survival of the organism. The transport network consists of three forces: open, narrow and closed. Desiring cells obtain sufficient adhesion molecules in each of these three directions, leading to the emergence of working mechanisms and groups within the transport network.\newline%
The experiments of Kevin Korsma of Children's Hospital San Diego showed that the cancer cell can store and protect these adhesion molecules as they grow.\newline%
During an interstitial stage of cancer cell growth, such as metastasis, metagenes are broken up and developed into unique individual adhesion molecules called minocyclins. These minocyclins allow the cell to grow under an optimal degree of separation from its host cell and to retain active adhesion forces in addition to forming a strong affinity within its outer layers for the host cell.\newline%
In follow{-}up experiments, Korsma is showing that the interstitial stage of cancer cell death is reduced by BCG, and the accumulation of the species fragments consisting of metagenes is diminished by BCG.\newline%
The metabolism of metagenes, particularly microscopic minocyclins, is essential for the survival of the patient. Though phenylephrine and valin might be used to mitigate the effects of BCG, these have a major limitation as well as the differences in side effects associated with phenylephrine and valin. Due to these difficulties, a unique strategy has been devised to support the positive effect of BCG therapy on cancer cells in the daily therapy routine.\newline%
Korsma is proposing a novel way to acquire this broad spectrum of Metagenos, as well as new antibody targeting variations in an ordinary cytotoxic regimen. In mice with metastatic(ES) non{-}small cell lung cancer and prostate cancer (PTCL), BCG has demonstrated an anti{-}tumor effect by reducing cancer cell resistance to the treatment.\newline%
In development, Korsma has developed a new peptide of Metagenos that, when coupled with BCG, will be capable of generating anti{-}tumor properties in the tumor. These agents, Korsma says, may be of optimal clinical significance for other cancers as well as non{-}small cell lung cancer.\newline%
There are currently no comparable metageno products available on the market.\newline%
The trial protocol stipulates the course of the study. Participating animals (treated with the mesenchymal precursor compound (MMP) of Metagenos, the protein{-}based metagenomic polymer (MMPP) of Metagenos, an approach based on data from other studies), will receive the therapy during the general admission clinic followed by a 24{-}hour, minimum treatment period of 24 hours in a non{-}surgical fashion followed by a week of regimen. Patients will be monitored during the treatment period. Three randomized patients will each receive eight (8) doses of BCG, a BCG delivery system called HIRT{-}2, a BCG delivery system called HTH{-}2 and Metagenos over one week. After three months of treatment, the last patient will receive 11 (11) doses of BCG, and the remaining patients will receive a regimen including HTH{-}2 and Metagenos. Only after the six{-}month study is complete will Metagenos be used for additional studies. In September 2004, Korsma intends to initiate a Phase II clinical trial (finding direct therapeutic effect) with a group of unidentified, cleared prostate cancer patients in which BCG will be treated with Metagenos with HHT{-}2 plus Metagenos.\newline%
BCG, kinase enzymes, alpha and gamma synthase, and se

%
\subsection{Image Analysis}%
\label{subsec:ImageAnalysis}%


\begin{figure}[h!]%
\centering%
\includegraphics[width=150px]{500_fake_images/samples_5_208.png}%
\caption{A Man With A Beard Wearing A Tie}%
\end{figure}

%
\end{document}