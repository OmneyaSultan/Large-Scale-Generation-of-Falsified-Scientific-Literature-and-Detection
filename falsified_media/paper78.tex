\documentclass{article}%
\usepackage[T1]{fontenc}%
\usepackage[utf8]{inputenc}%
\usepackage{lmodern}%
\usepackage{textcomp}%
\usepackage{lastpage}%
\usepackage{authblk}%
\usepackage{graphicx}%
%
\title{Epsilon{-}Toxin Production by Clostridium perfringens Type D Strain CN3718 Is Dependent upon the agr Operon but Not the VirS/VirR Two{-}Component Regulatory System}%
\author{Linda Stone}%
\affil{Institute of Neurological Sciences and Psychiatry, Hacettepe University, Ankara 06100, Turkey.}%
\date{01{-}01{-}2014}%
%
\begin{document}%
\normalsize%
\maketitle%
\section{Abstract}%
\label{sec:Abstract}%
SAN DIEGO (CNS) {-} Both sanitizers used at the San Diego Convention Center on Tuesday are now being tested by the Centers for Disease Control and Prevention for possible linkages to Daphropsia A{-}1 International (DASI) of the City of San Diego.\newline%
On Monday, San Diego city health officials said a batch of pasteurized water at the convention center contained DASI, the second{-}highest level of DASI detected at the hotel this year.\newline%
The city learned it could have a connection to specific pathogens, such as Black Gardasil virus, when a wastewater circuit breaker detected E. coli 1, 2 and 3 at the convention center's DPBS (DPW), San Diego Department of Public Health (SDDPH) reported.\newline%
"The possibility of an environmental contaminant is extremely rare, and very unlikely to occur at any convention facility," said Dr. Mario C. Diaz, director of public health services at SDDPH. "However, it would be a concerning event if the concerns are proven to be true."\newline%
The DPBS is scheduled to shut down for one day on Tuesday after health officials confirmed the presence of E. coli 1 and 2 in water handled by Pacific Water Services' General Wastewater operations.\newline%
Pacific Water Services is the lead wastewater operator at the San Diego Convention Center, which is attended by thousands of convention attendees each year.\newline%
Officials with Pacific Water Services and the SDDPH couldn't immediately be reached for comment.\newline%
On New Year's Day, Pacific Water Services issued an alert to the public, saying it had not been notified about any laboratory diagnoses and would continue to monitor patients at the convention center. Pacific Water Services agreed to submit additional testing for the Westine positive for E. coli 1 to the CDC, which will take up to three weeks.\newline%
The CDC has so far reported four cases of E. coli 1 and 2 from San Diego County through August 31, 2014. The CDC advises that people wash their hands thoroughly with soap and water when preparing food or handling any raw milk or other food that may contain E. coli. E. coli can cause severe dehydration and infections in young children, elderly people and people with weakened immune systems.\newline%
San Diego County health officials today expanded their presence in Southwest and Southwest Riverside Counties. It was revealed Wednesday that health officials have expanded their presence in La Jolla to include crews sifting through soil contaminated with QT 28, a strain of Escherichia coli O157:H7 known to cause serious infections in individuals undergoing chemotherapy. It was also revealed Wednesday that medical experts in San Diego have tested and found positive for E. coli O157:H7 that is circulating in milk from City of La Jolla.

%
\subsection{Image Analysis}%
\label{subsec:ImageAnalysis}%


\begin{figure}[h!]%
\centering%
\includegraphics[width=150px]{500_fake_images/samples_5_17.png}%
\caption{A Close Up Of A Person Wearing A Suit And Tie}%
\end{figure}

%
\end{document}