\documentclass{article}%
\usepackage[T1]{fontenc}%
\usepackage[utf8]{inputenc}%
\usepackage{lmodern}%
\usepackage{textcomp}%
\usepackage{lastpage}%
\usepackage{authblk}%
\usepackage{graphicx}%
%
\title{\_\_{-}Actinin TvACTN3 of Trichomonas vaginalis Is an RNA{-}Binding Protein That Could Participate in Its Posttranscriptional Iron Regulatory Mechanism}%
\author{Angela Sullivan}%
\affil{Department of Biochemistry, Osmania University, Hyderabad, A.P., India}%
\date{01{-}01{-}2014}%
%
\begin{document}%
\normalsize%
\maketitle%
\section{Abstract}%
\label{sec:Abstract}%
By Catherine Hagan, Rebecca Dorfman, Heather Sparks\newline%
If you are interested in menstrual relief, temporary muscle stiffness, or other conditions that might arise when you have an extreme period, you could do worse than seek out several of these anti{-}inflammatory drugs. But for non{-}muscle itching, jaundice, and rashes, the effects of the drugs are limited. Now researchers at Boston University have discovered that one of the first drugs (Falealefaremaotron) is able to identify a protein that contributes to the way types of interferon fare in the blood, suggesting that the HbA1c receptor plays a major role in the levels of these medications in the body. The findings suggest that future treatments will need to focus on targeting the HbA1c receptor to provide longer{-}lasting response.\newline%
While studies are still ongoing to determine the full and potentially useful effects of the discovery, the possible impact of this discovery could be extremely beneficial for patients who suffer from problems such as pruritus, and who find that this medication has not achieved the desired effect.\newline%
Boston University College of Medicine News

%
\subsection{Image Analysis}%
\label{subsec:ImageAnalysis}%


\begin{figure}[h!]%
\centering%
\includegraphics[width=150px]{500_fake_images/samples_5_139.png}%
\caption{A Man In A Suit And Tie Holding A Camera}%
\end{figure}

%
\end{document}