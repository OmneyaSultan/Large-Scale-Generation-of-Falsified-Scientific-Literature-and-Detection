\documentclass{article}%
\usepackage[T1]{fontenc}%
\usepackage[utf8]{inputenc}%
\usepackage{lmodern}%
\usepackage{textcomp}%
\usepackage{lastpage}%
\usepackage{authblk}%
\usepackage{graphicx}%
%
\title{Distinct expression of C4.4A in colorectal cancer detected by different antibodies}%
\author{Patrick Jackson}%
\affil{Second Department of Internal Medicine, Tottori University School of Medicine, Tottori 683{-}8504, Japan}%
\date{01{-}01{-}2012}%
%
\begin{document}%
\normalsize%
\maketitle%
\section{Abstract}%
\label{sec:Abstract}%
An experimental microtubule{-}targeting agent specifically designed to lower treatment response in patients with high grade and early breast cancer has successfully been compared with another microtubule{-}targeting agent that is targeted at the microtubule{-}cortical axis (MCT), researchers reported today.\newline%
Methylation of the microtubule{-}targeting agent docetaxel in patients with a range of aggressive and deadly forms of triple negative breast cancer (TRC) was driven by a significant reduction in metastatic treatment response and showed that MCT{-}targeting anti{-}cancer compounds specifically from docetaxel worked independently of MCT to induce complete and partial remission of tumor with no functional response during active tumor activity, according to a paper presented at the CTRC{-}AACR San Antonio Breast Cancer Symposium.\newline%
Resistance to advanced treatment with RASSF1A was even less observed in young, resistant patients compared with those who previously received docetaxel.\newline%
The study findings suggest that RASSF1A may be an effective prophylactic therapy in low{-} to intermediate{-}risk patients with tumor characteristics similar to that of advanced cancer.\newline%
Methylation of MCT{-}targeting anti{-}cancer compounds, along with new and personalized chemotherapeutic methods to selectively modulate the expression of MCT{-}targeting cytotoxic anti{-}tumor cells, are urgently needed to enable the latest developments in cancer treatment, said the studys principal investigator, Henry Sperling, MD, of Stanford University School of Medicine.\newline%
The development of treatments that reduce tumor response and induce complete and partial remission of a patients metastatic breast cancer is essential to further strengthen the global treatment efforts to tackle the most prevalent forms of breast cancer, said Sperling.\newline%
Sperlings research team compared RASSF1A with certain toxicants of European breast cancer therapies, such as docetaxel, cisplatin and a malignancy{-}remitting anti{-}HER2 (KIT) agent, which have failed to produce the desired effect in the human breast cancer population. In a first{-}of{-}its{-}kind comparison of MCT{-}targeting (c> aminostatin) with compounds of European breast cancer drugs, methylation (c> algenpantucel{-}L), a drug being developed for non{-}nodular non{-}Hodgkin lymphoma, showed clear and specific inhibition of cytotoxic tumor{-}specific MCT{-}targeting anti{-}mucosolid microtubule ligands (CTLGs) and reduced tumor metastases as tumor recurrence.\newline%
Related Content

%
\subsection{Image Analysis}%
\label{subsec:ImageAnalysis}%


\begin{figure}[h!]%
\centering%
\includegraphics[width=150px]{500_fake_images/samples_5_468.png}%
\caption{A Man In A Suit And Tie Is Smiling}%
\end{figure}

%
\end{document}