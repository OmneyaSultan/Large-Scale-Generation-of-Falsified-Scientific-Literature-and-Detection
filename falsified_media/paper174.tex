\documentclass{article}%
\usepackage[T1]{fontenc}%
\usepackage[utf8]{inputenc}%
\usepackage{lmodern}%
\usepackage{textcomp}%
\usepackage{lastpage}%
\usepackage{authblk}%
\usepackage{graphicx}%
%
\title{Fine Specificity of Plasmodium vivax Duffy Binding Protein Binding Engagement of the Duffy Antigen on Human Erythrocytes}%
\author{Barbara Parker}%
\affil{INSERM, U895 (quipe 1), Equipe lablise Ligue Contre le Cancer, C3M, 06204 Nice, France}%
\date{01{-}01{-}2013}%
%
\begin{document}%
\normalsize%
\maketitle%
\section{Abstract}%
\label{sec:Abstract}%
Doctors and scientists in the San Diego area are starting to answer a question: Are autoimmune disease more common in people who have the Type 2 diabetes type 2, or do people born with the disease carry more than normal levels of an inflammatory mediator in their blood and immune systems?\newline%
It is a question that has intrigued mental health researchers in the past. They have been searching for answers for years.\newline%
In the meantime, the U.S. Centers for Disease Control and Prevention reports that blood in your blood tests can signal a person's body is harboring chronic and aggressive autoimmune diseases like rheumatoid arthritis, rheumatoid{-}induced arthritis and lupus, among other illnesses.\newline%
One way the body is designed to keep itself healthy is its ability to stop chronic inflammation from going on.\newline%
The "natural immune system" helps keep the body focused on its duties, allowing us to survive and heal.\newline%
But people born with chronic autoimmune diseases have the ability to produce inflammatory mediators in their blood that alter the way our immune system recognizes and responds to infectious agents.\newline%
The increase in the levels of these inflammatory mediators is responsible for some of the symptoms of inflammatory diseases such as rheumatoid arthritis, the better known autoimmune disease of the first part of her body, and lupus, where the immune system reacts to an unexpected stimulus that causes or exacerbates inflammation.\newline%
One study, appearing in Nature Reviews Medical Physics, suggests that people born with Type 2 diabetes have slightly higher levels of the systemic inflammatory mediators than people who are healthy.\newline%
This study also suggests the immune system's reaction to new viral and infectious disease is similar to that it would have had a normal immune response in a person born with Type 2 diabetes.\newline%
Dr. Matthew Krebs, a clinical researcher who is working to understand the health implications of these immune system changes, said he also suspects these new findings could explain why rheumatoid arthritis is less common in people with blood levels at higher levels of the active inflammatory mediators.\newline%
While recent studies have found that higher blood levels of multiple inflammatory mediators in the body may trigger a strong immune response to infection and blood clots, a further study looking at the affects of autoimmune disease and the immune system was a challenge.\newline%
"One of the things we were excited about the first study was to find different patients and see how they respond to different interactions in the system," said Dr. Matthew H. Faraday, a senior investigator for the National Multiple Sclerosis Society in San Diego.\newline%
It turns out, he said, that the type of immune system that responds to infections in the body also affects levels of immune mediators.\newline%
"We wanted to find out what levels were responsible for the cycle that we see in chronic inflammatory diseases," he said.\newline%
Some patients found a way to control their immune system. Another study found that astrocytes, an immune cell that attaches to the surface of invaders, also sought to suppress chronic inflammation by generating higher levels of cell{-}killing protein, called NET{-}BLANB.\newline%
Dr. Teresa Madigan, an assistant professor of medicine at the UC San Diego School of Medicine, said the immune system often sees changes in its own cells in various ways. But she noted that many of the same cells in the blood cell structure could be involved in activating inflammatory mediators.\newline%
The findings of the experiments being reported in Nature Letters challenge a recent consensus that autoimmune diseases are increasing more in the United States than other countries.\newline%
Dr. Robert Butler, a professor of medicine at UCSD, said his department recently started a study aimed at establishing the actual cause of the disease, while testing the effects of treatment and, ultimately, prevention.\newline%
In the meantime, he said, patients often feel the need to make necessary adjustments to their lifestyles to fight for their health.\newline%
"We are desperately trying to figure out what causes these autoimmune diseases and how to treat them," he said.

%
\subsection{Image Analysis}%
\label{subsec:ImageAnalysis}%


\begin{figure}[h!]%
\centering%
\includegraphics[width=150px]{500_fake_images/samples_5_256.png}%
\caption{A Black And White Photo Of A Black And White Background}%
\end{figure}

%
\end{document}