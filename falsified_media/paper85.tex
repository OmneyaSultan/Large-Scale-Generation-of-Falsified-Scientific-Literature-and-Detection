\documentclass{article}%
\usepackage[T1]{fontenc}%
\usepackage[utf8]{inputenc}%
\usepackage{lmodern}%
\usepackage{textcomp}%
\usepackage{lastpage}%
\usepackage{authblk}%
\usepackage{graphicx}%
%
\title{Effect of sodium butyrate on lung vascular TNFSF15 (TL1A) expression: Differential expression patterns in pulmonary artery and microvascular endothelial cells}%
\author{Kevin Miller}%
\affil{Department of Orthopedic Surgery, Xinhua Hospital, Shanghai Jiaotong University, School of Medicine, Shanghai 200092, P.R. China}%
\date{01{-}01{-}2005}%
%
\begin{document}%
\normalsize%
\maketitle%
\section{Abstract}%
\label{sec:Abstract}%
A new study indicates that one enterococcus treatment for one species of bacteria in Europe is associated with a higher risk of infections by other species.\newline%
"CBD{-}Daprosine (VanD{-}Type glycopeptide{-}Resistant Enterococcus) infection of colonitis has been reported in humans for the first time," says Dr. Karel Shilgin, Fromada GSM Professor, Department of Infectious Diseases and Mid{-}West Branch Director, University of Pirbright in the UK. "These findings may lead to more intensive evaluation of primary prevention strategies and further development of new technology to protect humans from secondhand drugs."\newline%
Numerous studies show that research{-}based drugs are ineffective when used on live animals for routine human infection. Treatment guidelines have become geared toward better protective measures such as clean water, improved sanitation, cleaning and antimicrobial agents. In contrast, DDEs (VanD{-}Type glycopeptide{-}Resistant Enterococcus) are routinely applied in animal models and laboratory animal models are designed for select patients. However, these therapies are often ineffective in humans.\newline%
"Dope{-}Daprosine (VanD{-}Type glycopeptide{-}Resistant Enterococcus) was not developed as a drug for humans and did not meet rigorous international standards until biologics were introduced in the late 1990s," says Dr. Shilgin. "It had demonstrated adequate efficacy in reducing infections among dogs, cats and some rodents when used as a drug to treat constipation. Researchers at Ata, Vanderbilt, Korea, \& Hopkins responded to this clinical need and have been developing and introducing a new drug for malaria."\newline%
"Many DDEs are effective when given to enteric polices, which also contain various anti{-}microbial agents. But the major drawback is the risk of deadly bacteria.\newline%
"DDEs treat bugs both in humans and in animals and with occasional exposure to them people can develop infections."\newline%
Source: The Osteopathic Medical Association and Medical University of Glasgow

%
\subsection{Image Analysis}%
\label{subsec:ImageAnalysis}%


\begin{figure}[h!]%
\centering%
\includegraphics[width=150px]{500_fake_images/samples_5_176.png}%
\caption{A Black And White Photo Of A Black And White Cat}%
\end{figure}

%
\end{document}