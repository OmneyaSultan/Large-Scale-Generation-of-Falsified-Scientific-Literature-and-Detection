\documentclass{article}%
\usepackage[T1]{fontenc}%
\usepackage[utf8]{inputenc}%
\usepackage{lmodern}%
\usepackage{textcomp}%
\usepackage{lastpage}%
\usepackage{authblk}%
\usepackage{graphicx}%
%
\title{Loss of the Par3 Polarity Protein Promotes Breast Tumorigenesis and Metastasis}%
\author{Crystal Barry}%
\affil{CENAR and Department of Molecular Medicine, Faculty of Medicine, University of Malaya, Kuala Lumpur, Malaysia}%
\date{01{-}01{-}2014}%
%
\begin{document}%
\normalsize%
\maketitle%
\section{Abstract}%
\label{sec:Abstract}%
In this CRP B review study, researchers modify protein molecules from a region of the gene based on specific mutations in SCD to allow the human cells to express their dominant mutations using modified three{-}dimensional filaments.\newline%
CRP B specifically serves as a marker for C2s expression and SCD expression. This provides bioinformatics opportunities to identify gene expression in cancer cells and to investigate the relationships between positive and negative variations in CRP B and cancer.\newline%
Data show that CSR9 (a protein produced by ACS/ALCHAM in CA15 and INL1) accelerated C2 expression and the cycle length of related SCD mutations within existing C2 genes. While CRP is an important biochemical marker for C2 expression in cancer, it may not be applicable to test for the sensitivity of cancer screening in humans. In theory, studying the unique array of CRP molecules in SCD could provide the information that is otherwise missing from laboratory and clinical tests.\newline%
The most prevalent cancer in people with a mutation in CA15 or INL1 (J2342) is leukemia, which accounts for 25\% of all cancers. Being closely associated with tumors that progress rapidly and aggressive, there is a need for a clinically relevant test for selected patients.\newline%
The leading testing for CRP in cancer patients is type A negative, as associated with several cancers such as breast, ovarian, bowel, lupus, prostate, and skin. Common mutations in C1s or CA1s account for only one to three\% of the CA15 or INL1 mutation score of nearly 90\% of cancers. The human cell cycle length (longitudinally) of cancer cells, which has an effect on the progression of the tumor, varies in most CA15/INL1 mutation variants. To determine the relative sensitivity of the observed polymorphisms and predicted gene expression, individualized trials were conducted in individuals with complete CA15/INL1 mutation scores.\newline%
The authors showed that patients with variants associated with aggressive cancers displayed considerably higher genetic susceptibility to carcinogenesis in common cancer types in comparison to patients with normal mutation scores. If these cancer cells are exposed to the highest severity of CA15/INL1 mutations in a clinical setting, the likelihood of carcinogenesis in these cancer types is amplified. The authors found that this finding suggested higher C2 expression for C1s in J2342 and CA15/INL1 variants and more cancer aggressiveness in CA15/INL1 cancers compared to patients with traditional CA1/INL1 models of cancer. Interestingly, however, the disease{-}control model that was used to perform the study was of intermediate importance in terms of the function of the latest generation CRP translational databases (W3D/Ns) in identifying associated C2 expression for C1s and CA1s and does not allow you to explore the functional relevance of traditional CRP.\newline%
The study demonstrated that broad range of gene expression can be obtained from transcription factors generated by ACS/ALCHAM in cancer.\newline%
*Abstract:\newline%
1. CIRC: Topical Chaperone Test for Biological Aging in Cancers of the Heating Inflammatory Components and Acute Arteries. (Department of Chemistry)\newline%
2732111

%
\subsection{Image Analysis}%
\label{subsec:ImageAnalysis}%


\begin{figure}[h!]%
\centering%
\includegraphics[width=150px]{500_fake_images/samples_5_304.png}%
\caption{A Man In A Suit And Tie Holding A Toothbrush}%
\end{figure}

%
\end{document}