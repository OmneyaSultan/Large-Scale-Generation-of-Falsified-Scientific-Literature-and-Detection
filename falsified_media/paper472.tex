\documentclass{article}%
\usepackage[T1]{fontenc}%
\usepackage[utf8]{inputenc}%
\usepackage{lmodern}%
\usepackage{textcomp}%
\usepackage{lastpage}%
\usepackage{authblk}%
\usepackage{graphicx}%
%
\title{Anti{-}inflammatory effect of aldose reductase inhibition in murine polymicrobial sepsis}%
\author{Elizabeth Washington}%
\affil{Department of Biochemistry, Osmania University, Hyderabad, A.P., India}%
\date{01{-}01{-}2014}%
%
\begin{document}%
\normalsize%
\maketitle%
\section{Abstract}%
\label{sec:Abstract}%
Human Genome{-}Wide RNAi Screen Identifies an Essential Role for Inositol Pyrophosphates in Type{-}I Interferon Response\newline%
Manhasset, NY (January 1, 2014) {-} On December 18, 2014, researchers at the Society for Neuroscience and Massachusetts General Hospital (MGH) announced that human genetic information from a large EIRP (Evaluate Inositol Transcription Factor) cell pasting procedure had been matched to a large database of over 13,000 published toxicology reports in the peer{-}reviewed scientific literature, and predicted correctly that a protein related to a naturally occurring target in type{-}I interferon and ribosomal complement is at least part of the mechanism for the receptor specific a PK (random selected target receptor). This precision coding represents a validation of therapeutic approaches to clinical use of this promising class of compounds in human cultures with complex fracture stress and lesion repair.\newline%
A paper describing the research will be published in the January, 24th issue of the Clinical Antimicrobial Chemotherapy Journal.\newline%
This work validates previously published attempts to map the behavior of environmental and genetic markers that may predict which complexes and types of proteins would respond to drugs in humans, said Dr. Eshelman H. Adam, Director of the Institute for Metabolic and the Health Sciences of MGH. The group has shown that screening for regulatory proteins normally located downstream of interferon{-}A may be a promising way of determining which therapeutic targets would work best in animal models, because it suggests regulatory proteins associated with certain inflammatory and inflammatory{-}mediated reactions should be evaluated for specific therapeutic targets outside of normal clinical mouse models.\newline%
The Integrated Biological Expression Based Selection (IBES) testing using human genome{-}wide

%
\subsection{Image Analysis}%
\label{subsec:ImageAnalysis}%


\begin{figure}[h!]%
\centering%
\includegraphics[width=150px]{500_fake_images/samples_5_74.png}%
\caption{A Black And White Photo Of A Black And White Cat}%
\end{figure}

%
\end{document}