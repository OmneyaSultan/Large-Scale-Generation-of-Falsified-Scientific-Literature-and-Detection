\documentclass{article}%
\usepackage[T1]{fontenc}%
\usepackage[utf8]{inputenc}%
\usepackage{lmodern}%
\usepackage{textcomp}%
\usepackage{lastpage}%
\usepackage{authblk}%
\usepackage{graphicx}%
%
\title{Characterization of a Large Outbreak by CTX{-}M{-}1{-}Producing Klebsiella pneumoniae and Mechanisms Leading to In Vivo Carbapenem Resistance Development}%
\author{Kathryn Wood}%
\affil{Department of Cardiology, Zhongda Hospital, Medical School of Southeast University, Nanjing, Jiangsu, China}%
\date{01{-}01{-}2013}%
%
\begin{document}%
\normalsize%
\maketitle%
\section{Abstract}%
\label{sec:Abstract}%
SCIENTISTS from the University of California, San Diego School of Medicine have identified a molecular mechanism through which Huntingtons disease might facilitate breast cancer cell migration and invasion.\newline%
The research team, which included Dr. Ariane Quimpo, instructor of pharmacology, discovers that thyroid hormone (ThyroTHTOR) and non{-}thyrodermic hormone (HIF1a) support the development of tumor cell production and migration in the human breast cancer.\newline%
Published in the International Journal of Molecular Medicine, the research paper provides new insight into how the thyroid and other non{-}thyrodermic hormone drive breast cancer cell migration. The researchers also identify a molecular pathway for HIF1as self{-}defense against breast cancer cell invasion.\newline%
We understand that hormone producing T{-}cells are distinguished by their role in the defense of the thyroid gland against tumor cells in healthy and tumor{-}associated tissues, says Dr. Quimpo. In Huntingtons disease, hormone disrupting T{-}cells allow the tumor cells to flee the thyroid gland and spread outward. We are finding a molecular mechanism that actively fights tumour cell proliferation in breast cancer cell types characterized by HIF1a formation and migration, which is in line with Huntingtons disease that has been observed in multiple tumor samples.\newline%
In terms of Huntingtons disease where thyroid function is very poor in certain breast cancers, the trend toward high levels of thyroid stimulating hormones to enable tumor cell production and metastasis can be expected, says Dr. Adam Berman, Department of General Pharmacology, at the Stanford University School of Medicine. We saw this in Huntingtons disease only with the most aggressive forms. Thyroid stimulating hormones provide a common trigger for tumor cell growth in breast cancer cells with abundant tumor suppressor genes. This finding is important in understanding why thyroid stimulating hormones are so strong and attractive for tumor cells, and where the mechanisms underpinning it originate.\newline%
Researchers demonstrated that HIF1a breaks down the defenses of female thyroid T{-}cells which compete with free radicals on formation of tumor cells. The researchers also find evidence of insulin{-}like growth factor (IGF) activation. IGF is a key factor in pro{-}cancer stem cell formation and induced pluripotent stem cell (iPS) cell proliferation, both features of healthy breast cancer cells.\newline%
HIF1a is associated with enhanced cell migration, tumor proliferation, and metastasis. In addition, IGF has a significant anti{-}inflammatory effect on the human breast cancer immune system and in clinical studies has been found to suppress development of cancer stem cells and might impede the ability of tumor cells to spread.\newline%
We believe these findings may have direct clinical relevance with regards to breast cancer cell migration, Dr. Berman notes. Moreover, we believe that encouraging the development of breast cancer stem cells is the ultimate goal of the HIF1a program.\newline%
undefined

%
\subsection{Image Analysis}%
\label{subsec:ImageAnalysis}%


\begin{figure}[h!]%
\centering%
\includegraphics[width=150px]{500_fake_images/samples_5_491.png}%
\caption{A Close Up Of A Person Holding A Pair Of Scissors}%
\end{figure}

%
\end{document}