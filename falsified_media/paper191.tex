\documentclass{article}%
\usepackage[T1]{fontenc}%
\usepackage[utf8]{inputenc}%
\usepackage{lmodern}%
\usepackage{textcomp}%
\usepackage{lastpage}%
\usepackage{authblk}%
\usepackage{graphicx}%
%
\title{Ectopic Expression of a Maize Hybrid Down{-}Regulated Gene ZmARF25 Decreases Organ Size by Affecting Cellular Proliferation in Arabidopsis}%
\author{Patricia Jenkins}%
\affil{Division of Oncology/Hematology, Department of Internal Medicine, Korea University College of Medicine, Seoul, Republic of Korea, Division of Oncology/Hematology, Department of Pathology, Korea University College of Medicine, Seoul, Republic of Korea, Division of Oncology/Hematology, Department of Radiology, Korea University College of Medicine, Seoul, Republic of Korea, Division of Oncology/Hematology, Department of Surgery, Korea University College of Medicine, Seoul, Republic of Korea, Department of Physiology, College of Medicine, Hanyang University, Seoul, Republic of Korea}%
\date{01{-}01{-}2014}%
%
\begin{document}%
\normalsize%
\maketitle%
\section{Abstract}%
\label{sec:Abstract}%
The individual mutated nucleolin gene is implicated in the formation of malignant tumors, which is why researchers uncovered new tumor hotspots and explored new therapeutic targets for its co{-}protective role in cancer. Specially for BP level disease, the nucleolin gene is overexpressed in rare human models and associated forms of solid tumors. However, the enzyme also has potential clinical applications in an aggressive form of breast cancer and in its antimicrobial role in solid tumors and bacterial{-}associated dysentery. Patients with HBP{-}dependent tumor malignancies were more sensitive to the inhibitor of E1147, a member of the family of cell{-}selective kinase 1 inhibitors (TCKIs). Translational genomics techniques revealed that the antisense PCSK9 molecule acts to block the production of nucleolin by blocking the gene that carries out the mRNA signaling necessary for proliferation. Also, several drugs were shown to have localized effects on tumor tissue at higher concentrations. Thus, these novel therapeutics are exciting targets for personalized medicine, enabling scientists to identify patient{-}specific compounds for each form of cancer.\newline%
E1147/Hp23TxKI/hp23ATC was defined as an inhibitor of E1147/Hp23TxKI/hp23ATC signaling kinase (ATK). The protein differs from Hp{-}212 of the field and was more potent at inducing somatic cell death and metastasis than Hp{-}2H2. E1147/Hp23TxKI/hp23ATC signaling kinase targeted in the last stage of cellular protein manufacturing (MPS2) has been viewed as a potential therapeutic target. Among the researchers, Estelle Rechter from University Hospital of Erlangen{-}Nuremberg (ECeM), Gothenburg University Hospital (GZH), University of Copenhagen (UTC), Stockholm University Hospital (STH), and the University of Bremen (IPL) were the authors.

%
\subsection{Image Analysis}%
\label{subsec:ImageAnalysis}%


\begin{figure}[h!]%
\centering%
\includegraphics[width=150px]{500_fake_images/samples_5_271.png}%
\caption{A Close Up Of A Person Wearing A Tie}%
\end{figure}

%
\end{document}