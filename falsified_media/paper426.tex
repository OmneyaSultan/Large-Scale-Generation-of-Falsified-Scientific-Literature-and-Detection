\documentclass{article}%
\usepackage[T1]{fontenc}%
\usepackage[utf8]{inputenc}%
\usepackage{lmodern}%
\usepackage{textcomp}%
\usepackage{lastpage}%
\usepackage{authblk}%
\usepackage{graphicx}%
%
\title{Pathological Impact of Hepatitis B Virus Surface Proteins on the Liver Is Associated with the Host Genetic Background}%
\author{Joseph Snyder}%
\affil{Institute of Neurological Sciences and Psychiatry, Hacettepe University, Ankara 06100, Turkey.}%
\date{01{-}01{-}2014}%
%
\begin{document}%
\normalsize%
\maketitle%
\section{Abstract}%
\label{sec:Abstract}%
Interplay of mevalonate and Hippo pathways regulates RHAMM transcription via YAP to modulate breast cancer cell motility (PROPoD)\newline%
PARSIPPANY, N.J. {-} Research in mice has found that changes in genes at a crucial juncture in breast cancer cells development, which occurs when they are as young as three years old, may lead to important changes in gene expression and resistance to drug therapy. The study, published in the Journal of Developmental Biology, was conducted by researchers at the Icahn School of Medicine at Mount Sinai in New York City and reported in Nature Communications.\newline%
Breast cancer is the most common type of cancer in women. The human papillomavirus (HPV) causes the majority of cancer cases, and it is the most common cause of sexually transmitted infection in the United States. Most cancer tumors do not produce new cancer genes, which makes it difficult to identify and target resistance to drug therapy. Breast cancer cells that develop at high levels of protein at a time, called cytomegalovirus{-}associated DNA (CMV{-}A) cells, can promote the creation of cancer with their large, dense mass, which limits the drug activity of the drug{-}resistant tumors.\newline%
The research at Mount Sinai not only revolutionized the evaluation of treatment options, it also increased the treatment efficiency of drug therapy for CMV{-}A{-}associated breast cancer, said study co{-}author Siva Vardanyan, Ph.D., assistant professor of pediatric oncology, Mount Sinai School of Medicine. Many physicians have identified tremendous breast cancer resistance because of these cancerous cells that have developed so large of an exposure to proteins that are known to play a role in regulating the tumors growth.\newline%
Researchers investigated this development using mice, where research tracks the development of CRISPR{-}Cas9 in cancer cells and deforms genes. The researchers used YAP  a gene{-}editing technology  to kill of non{-}CRISPR{-}targeting cancers, but not the CRISPR{-}specific ones. The results showed that the elimination of non{-}CRISPR{-}targeting cancer cells enhanced the cells survival in the study population. Researchers also studied the HER2 pathway, an enzyme on cancer cell mixtures that target non{-}CRISPR{-}targeting genes.\newline%
The study also shows that IGF{-}1 receptor (IGF1), the protein responsible for regulating breast cancer cell growth, and the progenitor cell, an important driver of cancer cell biology, could influence mammary carcinoma cell motility. In both cases, IGF{-}1 and IGF{-}1 activity had a negative impact on breast cancer cell maturation and motility. This shows that removing IGF{-}1 in breast cancer cell somatic stem cells may improve protection against progenitor cell growth that

%
\subsection{Image Analysis}%
\label{subsec:ImageAnalysis}%


\begin{figure}[h!]%
\centering%
\includegraphics[width=150px]{500_fake_images/samples_5_483.png}%
\caption{A Man In A Suit And Tie Is Smiling}%
\end{figure}

%
\end{document}