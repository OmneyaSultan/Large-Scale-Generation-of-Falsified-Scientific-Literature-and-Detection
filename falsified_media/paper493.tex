\documentclass{article}%
\usepackage[T1]{fontenc}%
\usepackage[utf8]{inputenc}%
\usepackage{lmodern}%
\usepackage{textcomp}%
\usepackage{lastpage}%
\usepackage{authblk}%
\usepackage{graphicx}%
%
\title{The effect of transforming growth factor \_1 on the crosstalk between autophagy and apoptosis in the annulus fibrosus cells under serum deprivation}%
\author{Jeremy Taylor}%
\affil{Stem Cell and Tissue Engineering Department, Research Center for Science and Technology in Medicine (RCSTiM), Tehran University of Medical Sciences, Tehran, Iran}%
\date{01{-}01{-}2011}%
%
\begin{document}%
\normalsize%
\maketitle%
\section{Abstract}%
\label{sec:Abstract}%
It is said that a protein which is kinetically identical to that produced by the flu virus may be its most effective method of creating a resistant virus and creates an easy safe vaccine.\newline%
In a study published in the New England Journal of Medicine (NEJM) authored by an international team of researchers, they demonstrate that a combination of the two proteins produced by the growing immuno{-}suppressant CD11c\_\_\_ CD8\_\_\_ Dendritic Cells, can enable the long{-}term protection of animals against a type of influenza virus that has proven itself impervious to other anti{-}viral vaccines.\newline%
A mixture of a CD11c\_\_\_ Dendritic Cells and the booster{-}antigenic CD8\_\_\_ Dendritic Cells, a recombinant antibody created with recombinant source material from a mouse gene{-}blasting axis, can now be produced when inserted into the bloodstream and given in doses sufficient to provide the immune responses required to protect animals from infection.\newline%
Our results with this antibody dual mode of delivery led the cross section of two human immune cells from the Hemagglutinin A sequence and CD11c\_\_\_ Dendritic Cells to both immuno{-}suppress and protect against the H5N1 influenza virus. The outcome was very encouraging, said head of the study identified as Associate Professor Jrgen Frunk from Maastricht University of Technology. The combination of two CD11c\_\_\_ Dendritic Cells with the booster antigenic CD8\_\_\_ Dendritic Cells appears to be capable of producing both immune responses to the influenza virus within a short time and would not require further therapy to enhance responses.\newline%
Dr. Kathleen K.J. Rutigliano from University of Arizona and Nicole Vedremans from University of Pennsylvania were the other authors of the study. In addition to their evaluation of this antibody, the authors produced a more advanced variant CD11c\_\_\_ Dendritic Cells with a much more potent anti{-}viral effect that were utilized in immune clearance of human pathogens.\newline%
The new results are published in NEJM.

%
\subsection{Image Analysis}%
\label{subsec:ImageAnalysis}%


\begin{figure}[h!]%
\centering%
\includegraphics[width=150px]{500_fake_images/samples_5_93.png}%
\caption{A Man With A Beard Wearing A Tie}%
\end{figure}

%
\end{document}