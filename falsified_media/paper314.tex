\documentclass{article}%
\usepackage[T1]{fontenc}%
\usepackage[utf8]{inputenc}%
\usepackage{lmodern}%
\usepackage{textcomp}%
\usepackage{lastpage}%
\usepackage{authblk}%
\usepackage{graphicx}%
%
\title{Fine Tuning Inflammation at the Front Door: Macrophage Complement Receptor 3{-}mediates Phagocytosis and Immune Suppression for Francisella tularensis}%
\author{Ashley Rice}%
\affil{Bellvitge Biomedical Research Institute (IDIBELL), Barcelona, Spain}%
\date{01{-}01{-}2012}%
%
\begin{document}%
\normalsize%
\maketitle%
\section{Abstract}%
\label{sec:Abstract}%
SAN DIEGO (KGTV) {-} A growing interest in herbicides has led to a study now investigating the use of herbicides to augment the immune defenses of this type of lung cancer cell.\newline%
Herbicides in today's arsenal typically include glyphosate, known to be a factor in human cancer cases, and the key ingredients in several of the herbicides, including Penta, Hybrid Acetate, Vibrio Northeplasia (VORB), and Guardolci THC.\newline%
Among a number of other research findings, Professor Jim McCormick, a member of the California cancer research organization Alimera Sciences, has found an increase in the Cinnamomea product used to inhibit CL1{-}5 motor proteins in human lung cancer cells.\newline%
When targetted to the ligands of the cells at high levels, it killed the most active CL1{-}5 protein in the cell and greatly enhanced uptake of the receptor which is critical for CL1{-}5 discovery.\newline%
Although Cinnamomea has been used as a therapy to fight Cli{-}Erlichocytosis in mice, it may be used as a broad spectrum agent to treat a wide range of other lung cancers including:\newline%
{-}Chronic Non{-}Hodgkin's lymphoma\newline%
{-}Hemorrhagic salinity of the blood (especially in clusters of known to be responsible for cirrhosis)\newline%
{-}Joint has been misdiagnosed by using other more powerful or well tolerated chemotherapies\newline%
{-}Chronic CIC is a rare disease in which tumors often quickly metastasize\newline%
{-}Long{-}term lung cancer patients may have, if not established, a pathology of chronic XRatennial Syndrome (XRS)\newline%
{-}Cox tumor cell carcinoma of the lung\newline%
{-}Lung carcinoma of the lung has been reported in people with early stage Chronic Lung Traumatology and Multiple Sclerosis (CLL).\newline%
A trial of this agent will now take place at the UC San Diego Cancer Institute to see whether it is capable of using either monoclonal antibodies or IL{-}4 monoclonal antibodies.\newline%
Rounding out the research findings, Professor McCormick found that Kilauea nephrologic carcinoma cases known to be fueled by ECPH increased by 20 percent when treated with the concentrated inhibitors.\newline%
Finally, Dr. Macara, who underwent liver and bladder surgery in 2006 for a painful condition of inflammation of the liver, expressed excitement about the findings.\newline%
I have always enjoyed experimental new and cutting edge research. I am very excited at how these results are being validated, and they provide potential for clinical development in this promising indication.\newline%
"CNLPH, Cystic Fibrosis and Chronic Liver Disease have all been favorable treatment areas for GALNS patients and the new findings around Cinnamomea demonstrate how the expanded immune regimen we are looking at could be improved in the treatment of these diseases by taking into account the new insights."

%
\subsection{Image Analysis}%
\label{subsec:ImageAnalysis}%


\begin{figure}[h!]%
\centering%
\includegraphics[width=150px]{500_fake_images/samples_5_382.png}%
\caption{A Black And White Photo Of A Zebra Standing In A Field}%
\end{figure}

%
\end{document}