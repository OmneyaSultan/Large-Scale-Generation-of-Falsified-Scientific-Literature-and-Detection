\documentclass{article}%
\usepackage[T1]{fontenc}%
\usepackage[utf8]{inputenc}%
\usepackage{lmodern}%
\usepackage{textcomp}%
\usepackage{lastpage}%
\usepackage{authblk}%
\usepackage{graphicx}%
%
\title{TspanC8 tetraspanins regulate ADAM10/Kuzbanian trafficking and promote Notch activation in flies and mammals}%
\author{Darrell Wilson}%
\affil{School of Medicine, Chung Shan Medical University, 110 Chien{-}Kuo N. Road, Section 1, Taichung 402, Taiwan}%
\date{01{-}01{-}1999}%
%
\begin{document}%
\normalsize%
\maketitle%
\section{Abstract}%
\label{sec:Abstract}%
Saying it was the "most important risk management decision made by a system operator," the advisory board agreed Friday to require GPS systems to help disabled people reach medical appointments or apply for unemployment and apply for disability benefits.\newline%
The five{-}year rule requires human{-}machine linkages in emergency situations in disaster areas across the United States. U.S. employers can claim at least two workers who use GPS systems {-}{-} one for the visually impaired and one for the mentally handicapped.\newline%
"Our hope is that this is going to make these small steps effective in making these systems more accessible to people with disabilities," said Sandra Husak, a spokeswoman for the National Association of Disabled Employees.\newline%
For many people, calling an emergency resource center or phoning a federally run service center can be daunting, said Diane Godfrey, legislative director for Washington State University Systems Engineering Technology division. Godfrey and director of the Douglas County staff of disability services, though, said the group agreed the GPS rule came too late.\newline%
"It is going to fall on deaf ears as far as we are concerned," she said.\newline%
Familiar identification\newline%
Using GPS, Americans could call a phone number to have someone to give directions or in emergencies, Godfrey said.\newline%
Companies have been considered potential beneficiaries of the GPS rule, even though only six states currently require such vehicles, Husak said. These companies include companies such as AT\&T Corp., who are mapping an inner half of the country and their employees will join up with devices that will pinpoint a person's location for more than 90 seconds.\newline%
Traveling with a GPS device gives companies another line of defense in case employees are disoriented by the GPS system.\newline%
"We are not necessarily saying we will not support this," said Jim Reichert, AT\&T's supervisor of satellite communications and activities. "But at the end of the day there have to be safeguards in place for these vehicles."\newline%
The same limitation applies to an insurance adjuster or mechanic using a GPS unit, Reichert said. He said the GPS system can be used "as a diagnostic tool."\newline%
"It's kind of like a key in a car," he said. "What are the rules around this?"\newline%
Reichert said AT\&T would also respond to the GPS rule in future computer system developments.\newline%
"If there is a violation, we will modify the software so that the GPS technology will conform with state law," he said.\newline%
Offbeat facts\newline%
The GPS rule joins other rules requiring inclusion in wireless equipment to assist the visually impaired and help deaf and hard of hearing people.\newline%
It is the second such bill on the subject this year. The\newline%
Act now requires federal financial assistance to many businesses that have satellite radio receivers or other alternatives to\newline%
Telephone, television or radio technology.\newline%
A similar law that has not been adopted so far in Congress is\newline%
seeking federal tax{-}funding from federal investment banks to help companies design and install the\newline%
devices.

%
\subsection{Image Analysis}%
\label{subsec:ImageAnalysis}%


\begin{figure}[h!]%
\centering%
\includegraphics[width=150px]{500_fake_images/samples_5_455.png}%
\caption{A Close Up Of A Person Wearing A Tie}%
\end{figure}

%
\end{document}