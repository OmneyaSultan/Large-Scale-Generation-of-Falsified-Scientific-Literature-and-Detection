\documentclass{article}%
\usepackage[T1]{fontenc}%
\usepackage[utf8]{inputenc}%
\usepackage{lmodern}%
\usepackage{textcomp}%
\usepackage{lastpage}%
\usepackage{authblk}%
\usepackage{graphicx}%
%
\title{Drosophila Mgr, a Prefoldin subunit cooperating with von Hippel Lindau to regulate tubulin stability}%
\author{Gary Roberson}%
\affil{Department of Surgery, University of Wisconsin Hospital and Clinics, Madison, Wisconsin, United States of America}%
\date{01{-}01{-}2013}%
%
\begin{document}%
\normalsize%
\maketitle%
\section{Abstract}%
\label{sec:Abstract}%
(Removing active bladder cancer cells from individual patients for five days without radiation is said to improve survival and survival at outpatient sites, but not perfect!)\newline%
Clinicians at the University of Illinois, Chicago, reported findings today that a new protein found in the human urothelial bladder cancer, called HtrA1, that acts as a blood pressure regulator may provide clues to breast cancer and HIV drugs that block the signal of HtrA1. HtrA1 is active in many tissues, including kidney and liver cells and in the blood vessels surrounding the urothelial lymph nodes.\newline%
The discovery of the HtrA1 protein is considered a potential biomarker for breast cancer in particular and HIV/AIDS in particular.\newline%
Results of a phase I study with HtrA1 confirmed a change in HtrA1 function during the initial colonoscopic colonoscopy with reducing normal white blood cell count. The results indicate that HtrA1 is potentially a new, differentiating diagnostic marker of high risk breast tumors.\newline%
HtrA1 is capable of regulating blood pressure in a way that compounds safety concerns. Previous studies performed using HtrA1 in pituitary gland tumors have not provided evidence that the fluid holds phosphate to regulate blood pressure. And certain bodily functions are more regulated by blood pressure in urine than in tissues other than the urothelial bladder.\newline%
"Even if the p{-}2 inhibitor Pfizer, known as Partosamform, is well tolerated, the HtrA1 has the potential to be a biomarker that captures angiogenesis and signals the presence of better treated lesions," said Chris Till, professor of medicine at the Medical Center of Illinois and senior author of the report published in the New England Journal of Medicine.\newline%
The study was funded by the David H. Koch Institute for Integrative Cancer Research.

%
\subsection{Image Analysis}%
\label{subsec:ImageAnalysis}%


\begin{figure}[h!]%
\centering%
\includegraphics[width=150px]{500_fake_images/samples_5_2.png}%
\caption{A Black And White Photo Of A Black And White Cat}%
\end{figure}

%
\end{document}