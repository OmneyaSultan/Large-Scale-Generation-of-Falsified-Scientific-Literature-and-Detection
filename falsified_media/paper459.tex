\documentclass{article}%
\usepackage[T1]{fontenc}%
\usepackage[utf8]{inputenc}%
\usepackage{lmodern}%
\usepackage{textcomp}%
\usepackage{lastpage}%
\usepackage{authblk}%
\usepackage{graphicx}%
%
\title{Meningococcal Porin PorB Prevents Cellular Apoptosis in a Toll{-}Like Receptor 2{-} and NF{-}\_\_B{-}Independent Manner\_\_}%
\author{Joshua Wilkins}%
\affil{Department of Surgery, Faculty of Medicine, School of Medicine, Kaohsiung Medical University, Kaohsiung, Taiwan}%
\date{01{-}01{-}2014}%
%
\begin{document}%
\normalsize%
\maketitle%
\section{Abstract}%
\label{sec:Abstract}%
CAPE TOWN, South Africa (Reuters) {-} Overfamiliarity was to blame for South Africas dramatic collapse to a 108{-}run defeat to Australia in the second test on Wednesday as aspirin aggravated injuries to key players, limiting their ability to recover for the third and final test in Cape Town.\newline%
Aaron Finch hit a 51{-}ball 64 but even this breakthrough stand was never enough as Australia launched a dangerous chase on the sixth day at University Oval but Chris Lynn lost his off{-}stump in the penultimate over and was caught by wicketkeeper Quinton de Kock off the next ball.\newline%
South Africa added just 19 runs to their overnight score before lunch but early bowler Ravi Ashwin (1{-}19) and paceman Jackson Bird (2{-}23) both removed Phillip Hughes, with the former edging a catch to wicketkeeper Quinton de Kock.\newline%
Opener AB de Villiers, not out on 16, had seemed set to continue but Warner stepped on to the crease and was bowled by fellow opener Quinton du Plessis.\newline%
The victory secured Australias first series win over a top{-}ranked side since 1984{-}85 and at this rate South Africa would be clear favorites to win the second test starting on Friday.\newline%
Ive got to be honest, it was worse than the first Test. Really thats what I was talking about. We were susceptible to being predictable and to taking wickets, du Plessis told reporters.\newline%
Both spinners were sent in from the penultimate ball and South Africa lost three players in quick succession. Lasith Malinga, in his test bowleg for just the third time, had Lynn caught in the slips with the former Yorkshire batsmans fourth ball and Hashim Amla was chopped on off by Bird, whose inswinging yorker had Amla, on leg stump.\newline%
Finch fell in the same over and James Faulkner (3{-}73) then showed some urgency to take the follow{-}on.\newline%
The first six deliveries of the Australia innings were hit for six and Faulkner was only reviewed for review when he was given out on three when his feet were in the air and it was ruled a duck.\newline%
Losing Amla for the fifth time in 10 test innings put South Africa in a fix after de Kock and du Plessis were dismissed before stumps.\newline%
Paceman Dale Steyn was dismissed four overs before tea, caught at gully by batsman David Warner for two. Shafiul Islam (3{-}58) then struck in his first over of the day, taking Australia to a breakthrough when seamer Smith was bowled leg{-}before by phyicalist Hashim Amla.\newline%
South Africa lost all seven wickets for 66 runs in a dramatic seventh session to be finally bowled out for 227 in their second innings.\newline%
Hashim Amlas fifty was belatedly vindicated but veteran AB de Villiers was also lucky to survive with a diving catch to Smith at gully.\newline%
Ryan McLaren followed suit to register the second leading test century but this was just a wasted opportunity and South Africa, whose focus was on holding their nerve with the bat, were left with only the last 16 overs to save the match.\newline%
(Editing by Alan Baldwin/Peter Rutherford)

%
\subsection{Image Analysis}%
\label{subsec:ImageAnalysis}%


\begin{figure}[h!]%
\centering%
\includegraphics[width=150px]{500_fake_images/samples_5_62.png}%
\caption{A Close Up Of A Person Wearing A Shirt And Tie}%
\end{figure}

%
\end{document}