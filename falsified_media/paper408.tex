\documentclass{article}%
\usepackage[T1]{fontenc}%
\usepackage[utf8]{inputenc}%
\usepackage{lmodern}%
\usepackage{textcomp}%
\usepackage{lastpage}%
\usepackage{authblk}%
\usepackage{graphicx}%
%
\title{Human Genome{-}Wide RNAi Screen Identifies an Essential Role for Inositol Pyrophosphates in Type{-}I Interferon Response}%
\author{Lindsey Sweeney}%
\affil{Institute of Pharmacology, Toxicology and Pharmacy, Ludwig{-}Maximilians{-}University, Munich, Germany}%
\date{01{-}01{-}2012}%
%
\begin{document}%
\normalsize%
\maketitle%
\section{Abstract}%
\label{sec:Abstract}%
San Diego researchers, led by Yale Professor Robert Langer, a co{-}lead author of the study published today in Cell Reports, published their study to examine inflammation during small cell lung cancer treatments and malignant glioma treatment.\newline%
It is possible that immunotherapy agents for lymphoma may suppress BMP{-}9 overexpression of BMP{-}9 through inhibition of PI3K/AKT, in order to reduce tumor growth and metastasis. This flies in the face of the idea that prior to these treatments, leukemia with its high concentration of BMP{-}9 was a treatable condition.\newline%
In the study, the researchers assessed BMP{-}9 overexpression of PD{-}L1 and the ability of chemotherapy agents to induce fusion of BMP{-}9 to PD{-}L1. BMP{-}9 is a tumor suppressor that cell lines have been attached to to express a high affinity for PD{-}L1. PD{-}L1 has been selected as a relevant target because it acts as a mediator of cell turnover. It has been shown to bind to levels of BMP{-}9 that are implicated in tumor suppressor genes.\newline%
We report the first, albeit very small, evidence that showing BMP{-}9 overexpression of patients with chronic lymphocytic leukemia (CLL) could stimulate cell differentiation and migration to tumors so they could be treated in an immunotherapy treatment, says Edward O. Gakmon, MD, of OConnor Cancer Center, Yale University School of Medicine, who was also a co{-}lead author of the study.\newline%
The scientists note that blocking BMP{-}9 may be attractive to patients with inflammation and early stage disease. A potent antibody{-}drug conjugate drug that targets and binds to BMP{-}9 could be a good strategy. Studies of this drug have been underway for about three years and have demonstrated its ability to inhibit BMP{-}9. Gakmon notes that the next step will be to see whether drugs that inhibit BMP{-}9 inhibit PD{-}L1. This compound might be a good drug to lower the intensity of current chemotherapy agents.\newline%
Drug that inhibits BMP{-}9, or one of the natural antibodies that inhibit it, is a proven way to stimulate cell differentiation for cell growth in tumors, says Kaleisi M. Motsum, MD, a member of Yales Helen DeVos Cancer Center, who also co{-}authored the study. It should not be unexpected if scientists find that a drug that inhibits BMP{-}9 and N{-}acetylcysteine inhibitors do likewise in tumor cells.\newline%
Although current treatment for inflammatory diseases, including rheumatoid arthritis, is limited to chemotherapy, Motsum notes that scientists want to see a balance of medications that may inhibit the activity of BMP{-}9 and other parts of the immune system that have been identified.\newline%
The scientists caution that, because their study used cells in a single cell culture cell line rather than tissue from a patient, it is possible that these cells could have lost some BMP{-}9 activity. Although the researchers work was published in Cell Reports, the researchers say they need to collect more time before collecting specific population of cells that may have been co{-}opted in their study. Motsum, OConnor, and Gakmon say that more work is needed to confirm that the mice that were genetically altered to exhibit BMP{-}9 overexpression by the chemotherapy agent have the human phenotype they studied in the study.\newline%
The project was supported by the National Institutes of Health, a grant from the Pfizer Charitable Trust for Research on New Threats to Society; a grant from the National Institute of Cancer Research; the Pediatric Cancer Research Center at The Childrens Hospital of Philadelphia; the Kimmel Cancer Center for Children in Philadelphia; the National Institutes of Health; the Human and Spinal Marrow Transplantation and Stem Cell Research Development Program; the April 2012

%
\subsection{Image Analysis}%
\label{subsec:ImageAnalysis}%


\begin{figure}[h!]%
\centering%
\includegraphics[width=150px]{500_fake_images/samples_5_467.png}%
\caption{A Close Up Of A Mirror With A Reflection Of A Mirror}%
\end{figure}

%
\end{document}