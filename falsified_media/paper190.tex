\documentclass{article}%
\usepackage[T1]{fontenc}%
\usepackage[utf8]{inputenc}%
\usepackage{lmodern}%
\usepackage{textcomp}%
\usepackage{lastpage}%
\usepackage{authblk}%
\usepackage{graphicx}%
%
\title{Genotyping and Phenotyping of Beta2{-}Toxigenic Clostridium perfringens Fecal Isolates Associated with Gastrointestinal Diseases in Piglets}%
\author{Douglas Cook}%
\affil{Department of Biology, Pamukkale University, Kinikli Campus, 20070 Denizli, Turkey}%
\date{01{-}01{-}2014}%
%
\begin{document}%
\normalsize%
\maketitle%
\section{Abstract}%
\label{sec:Abstract}%
Using this new tool, 70 percent of test patients with recurrent colorectal cancer who were previously on VEGF agonist therapy showed signs of remission.\newline%
The numbers are reminiscent of those used in the Bush{-}era study, which found a similar pattern in colorectal cancer metastasis but showed slightly greater activity (plus{-}2 percent). The study doesnt use randomized clinical trials to compare the results of other drugs in more recent history, but this study can be read to make the point: Our new generation of cancer drugs cannot, in general, successfully alter tumor metastasis.\newline%
Genotype 1 colorectal cancer is a particular problem area for VEGF agonists because survival rates are better found in patients with genotype 1. If the mutation is found early and treated aggressively, the disease can be slow and glomerular sclerosing cholangitis, a swelling of the liver that can cause inflammation. The disease can be cured with chemotherapy in late stages, but patients have a relatively high mortality rate, about 38 percent in the late stages of treatment.\newline%
Studies such as this study suggest many of our drug combinations may have had a chance to work, but getting them to work required enabling patients to express VEGF via methylation of their genes. Genomic Tects is now a solid method of doing this, and its making progress in this area.\newline%
In addition to increasing choice, the expanded use of Xtandi this year by Baxter makes it more feasible for researchers to test modified therapies for genotype 1 colorectal cancer, the next line of therapy. Currently, CT scans are done for genotype 1 and genotype 2 patients. But these studies will also include a much broader class of earlier{-}stage patients that might be receptive to a modified therapy.\newline%
Last fall, with the International Society for Womens Health at the forefront, we introduced the Fenway Personalized Medicine program, which equips doctors, researchers and patients with the tools and strategies to reduce cancer risk. Fenway has emerged as a model of patient{-}centered quality improvement (QPI) programs for patient care, especially in the field of translational cancer research.\newline%
Rituxan (Rituxan) is the main drug used to treat second{-}line, advanced, HER2+ Hodgkins disease (nerve tissue damage). The goal of the Fenway program is to help patients delay bone marrow transplantation and avoid unnecessary surgery. Rituxan is not only safer and more effective than chemo, but it also can be used as a first{-}line treatment and have an increased safety profile.\newline%
In the Bay Area, Fenway is a prominent area in the new generation of personalized medicine. Were busy focusing on heart disease, another major group of cancer patients. But were stepping forward in this area too, especially with our latest invention, the Fenway Personalized Medicine Academy, where were partnering with UCSF to provide a variety of workshops for clinicians and researchers.\newline%
We have many more exciting advances in the treatment of patients with early{-}stage disease in the pipeline and were committed to making Fenway the engine of new cures.\newline%
For more information, please visit Fenway.org.

%
\subsection{Image Analysis}%
\label{subsec:ImageAnalysis}%


\begin{figure}[h!]%
\centering%
\includegraphics[width=150px]{500_fake_images/samples_5_270.png}%
\caption{A Black And White Photo Of A Mirror}%
\end{figure}

%
\end{document}