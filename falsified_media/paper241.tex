\documentclass{article}%
\usepackage[T1]{fontenc}%
\usepackage[utf8]{inputenc}%
\usepackage{lmodern}%
\usepackage{textcomp}%
\usepackage{lastpage}%
\usepackage{authblk}%
\usepackage{graphicx}%
%
\title{Adenovirus{-}mediated overexpression of BMP{-}9 inhibits human osteosarcoma cell growth and migration through downregulation of the PI3K/AKT pathway}%
\author{Amy Smith}%
\affil{The Johns Hopkins Oncology Center, Program in Human Genetics, and The Howard Hughes Medical Institute, The Johns Hopkins University School of Medicine, 424 N. Bond Street, Baltimore, 21231, Maryland, USA}%
\date{01{-}01{-}2014}%
%
\begin{document}%
\normalsize%
\maketitle%
\section{Abstract}%
\label{sec:Abstract}%
An intraepithelial re{-}expression of core membrane{-}related proteins takes place during an acute kidney injury that causes the tissue to refract oxygenated blood to the recipients organ. These proteins act to release oxygen and part of the targeted cellular matrix is affected by endothelial skin repair. After the stage in vascular circulation where angiotensin II neutralizes endothelial matrix proteins, patients may experience classic symptoms of continued buildup of scar tissue that prolongs the patients time in the intensive care unit and requires a tracheotomy tube to attend to. Although in vitro studies provided insight into the cellular processes that are regulated by angiotensin II to repair tissue damage to the kidney, it has been difficult to distinguish the action of these proteins. Researchers have found that reactive proteins within core membrane proteins target endothelial DNA damage for a full role in sending signals from core membrane proteins to endothelial cells. These proteins interact with endothelial cell structure in two important ways, designing and reinforcing proteins to cause or inhibit damage to the endothelial cell. One of these proteins results in an overexpression of one more die setting of neighboring apoptosis proteins. These proteins promote the destruction of endothelial bone insulation. The other protein reflects the activity of angiotensin II to transform arterial blood through a transvertebral channel to blood stream to initiate the re{-}expression of core membrane proteins. In a clinical trial, extended exposure of endothelial skin repair protein in the blood of a low{-}level renal fibroscopy patient showed a marked absence of the toxic polypotassium peroxide{-}treated arterial blood after 12 weeks of monitoring. This study was published in The American Journal of Kidney Diseases

%
\subsection{Image Analysis}%
\label{subsec:ImageAnalysis}%


\begin{figure}[h!]%
\centering%
\includegraphics[width=150px]{500_fake_images/samples_5_316.png}%
\caption{A Black And White Photo Of A Black And White Cat}%
\end{figure}

%
\end{document}