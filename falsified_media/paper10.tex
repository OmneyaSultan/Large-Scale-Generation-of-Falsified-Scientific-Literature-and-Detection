\documentclass{article}%
\usepackage[T1]{fontenc}%
\usepackage[utf8]{inputenc}%
\usepackage{lmodern}%
\usepackage{textcomp}%
\usepackage{lastpage}%
\usepackage{authblk}%
\usepackage{graphicx}%
%
\title{Genotyping and Phenotyping of Beta2{-}Toxigenic Clostridium perfringens Fecal Isolates Associated with Gastrointestinal Diseases in Piglets}%
\author{Heather Shelton}%
\affil{Department of Microbiology, Laboratory of Mycotoxins and Toxigenic Fungi, University of So Paulo, So Paulo, So Paulo, Brazil}%
\date{01{-}01{-}2013}%
%
\begin{document}%
\normalsize%
\maketitle%
\section{Abstract}%
\label{sec:Abstract}%
When stromal cells are born, the embryonic cell nucleus that contains the progenitor cells appears to be completely devoid of any human DNA, and all DNA is unknown. A closer look at the reason for this lack of human DNA would be an indication that the gene encoding the engineered stem cells, known as CD51A, may undergo epigenetic regulation.\newline%
In a study on male mice, Laetrile, an immunological regulator of natural killer cells of leukemia, showed that CD51A may influence epigenetic activity of specific CD51A{-}regulated genes such as Pacaribot, MyD{-}02, RLA, Pacaribot2, MTC2, and Y (Creator1) that attach to DNA, found in DNA{-}affixed protein phosphatases. As a result, CD51A might alter the functioning of these CD51A{-}regulated genes so that the transplanted stem cells could express these genes.\newline%
These gene{-}defective genes accumulate in the blood of mice, which include kidney cells and other tissues. This list includes gene{-}deficient individuals (e.g., cells and tissues that do not express normal DNA.) The study also states that CD51A may act on similar genes on the brain, spinal cord, ovaries, and some other internal organs such as the liver.\newline%
The study, published in Novocure, a life science company, provides insight into the importance of epigenetic regulation to human stem cell development. It is estimated that between 80 to 90 percent of C{-}kit induced pluripotent stem cells are pluripotent for a variety of target diseases and diverse human malignancies, and that about a third of these cells are generated in vitro for functional tissue regeneration.\newline%
The cell line with the highest expression of the therapeutic signaling pathway miR{-}3 in its DNA was labeled Basetostatin100 (human JP) v. Linkartatin{-}4 (human rJ) v. Immune Assembly Hakonarson (commonly known as Ras{-}4(2) or demfilament{-}1) v. JAK1 (a protein expressed by photoreceptor cells associated with tumor formation, metastasis, and infection). Basetostatin100 and all the other activation pathways for anti{-}tumor response in the mice were removed, and it was determined that Basetostatin100, imidacloprid (a stress molecule maintained by cancer cells), and rShungpiru were the pathway controlling the activation of molecular functions that influence stem cell propagation. The cell line appeared to exhibit an increase in RLM1 protein expression within days of transplantation.\newline%
This cell line showed the highest expression of the therapeutic signaling pathway Iktoin{-}1a, which was found to be a candidate gene for restoration of the normal function of macrophages, a type of immune system cell. Iktoin{-}1a, also known as Iktoin{-}1a+(k/k), is a protein produced by bone marrow in response to chemotherapy and can be prescribed for patients with certain cancers. Iktoin{-}1a, although they constitute only a small portion of the stem cell proliferation, is expected to be an important mediator of the innate immune response.

%
\subsection{Image Analysis}%
\label{subsec:ImageAnalysis}%


\begin{figure}[h!]%
\centering%
\includegraphics[width=150px]{500_fake_images/samples_5_108.png}%
\caption{A Close Up Of A Black And White Cat In A Room}%
\end{figure}

%
\end{document}