\documentclass{article}%
\usepackage[T1]{fontenc}%
\usepackage[utf8]{inputenc}%
\usepackage{lmodern}%
\usepackage{textcomp}%
\usepackage{lastpage}%
\usepackage{authblk}%
\usepackage{graphicx}%
%
\title{Cloning and Expression of the 44{-}Kilodalton Major Outer Membrane Protein Gene of the Human Granulocytic Ehrlichiosis Agent and Application of the Recombinant Protein to Serodiagnosis}%
\author{Jonathan Moyer}%
\affil{Department of Biochemistry, Institute of Medical Sciences, Banaras Hindu University, Varanasi, India}%
\date{01{-}01{-}2004}%
%
\begin{document}%
\normalsize%
\maketitle%
\section{Abstract}%
\label{sec:Abstract}%
During my angiogenesis and development of a heart, myocytes have a protein called NCC{-}KCATPase. The protein is a key protein structure of myocytes and when activated it has been used by myopathies such as myocardial heart attack. NaC{-}KCATPase is the answer when it comes to angioplasty, a procedure designed to open a blocked artery in the chest. The work published in Diabetes Progress in December 2004 indicates that interleukin{-}1 b (IL{-}1b) played a key role in the growth of myocytes in patients with myocardial heart disease and diabetes.\newline%
According to the authors, these results support the use of interleukin{-}1b for angioplasty in patients with cardiac cardiac disease due to myocardial heart disease and type 1 diabetes.\newline%
A factor that has kept well{-}characterized heart disease from progressing to neurodegenerative diseases is the strength of interleukin{-}1b in the blood. An area scientists study, found interleukin{-}1b in blood cholesterol levels as the reason why the cells persist in regeneration of cells.\newline%
This finding led to IL{-}1b being used in cardiac pumping, cardiac valves, insulins, and cardiac cell killing, in patients with coronary artery disease.\newline%
The study showed the role of interleukin{-}1b in cell regulation in myocytes. This finding was a significant contributing factor to the eradication of myocardial die{-}off in some individuals with chronic coronary artery disease.\newline%
Scientists believe that IL{-}1b and the proteins target in the Cellular Protein Signature Task, or CDPT, played an important role in in the calcium inamen function and in the immune systems defense of the heart.\newline%
The pattern of premature myocardial die{-}off in cardiac myocytes is a feature of a buildup of calcium in the blood generated by calcification of myocardial artery plaque. Until now, less than 1\% of patients coronary blood was missing, but this was only the beginning. Now other clinical trials have shown that neovascularization or bypass heart valve replacement is more successful with patients at or above beta{-}blockers therapy.\newline%
Interleukin{-}1b supplementation with a high level of calcium in patients blood is believed to have a number of therapeutic effects, such as targeting low gene expression and slowing down the coronary events triggered by activation of chi{-}bur{-}alpha and chi{-}pEG protein.

%
\subsection{Image Analysis}%
\label{subsec:ImageAnalysis}%


\begin{figure}[h!]%
\centering%
\includegraphics[width=150px]{500_fake_images/samples_5_233.png}%
\caption{A Close Up Of A Black And White Photo Of A Vase}%
\end{figure}

%
\end{document}