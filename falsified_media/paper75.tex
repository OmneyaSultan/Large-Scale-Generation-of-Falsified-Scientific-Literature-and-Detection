\documentclass{article}%
\usepackage[T1]{fontenc}%
\usepackage[utf8]{inputenc}%
\usepackage{lmodern}%
\usepackage{textcomp}%
\usepackage{lastpage}%
\usepackage{authblk}%
\usepackage{graphicx}%
%
\title{Terminalia catappa Exerts Antimetastatic Effects on Hepatocellular Carcinoma through Transcriptional Inhibition of Matrix Metalloproteinase{-}9 by Modulating NF{-}\_B and AP{-}1 Activity}%
\author{Zachary Becker}%
\affil{Department of Biochemistry, Osmania University, Hyderabad, A.P., India}%
\date{01{-}01{-}2013}%
%
\begin{document}%
\normalsize%
\maketitle%
\section{Abstract}%
\label{sec:Abstract}%
Scientists have created a cool method for instantly blocking the Legionella pneumophila bacterium (Diceranch) from locomotive decontamination over a short period of time.\newline%
Levi A Rubin, the main author of a report on the realization was recently released on the International Center for Biomedical Innovation (ICBI) website. Rubin and his group came up with the lightest manufacturing technique in the world and able to get rid of the Legionella pneumophila on three kilometres of its infected limbs, which are then remotely sealed inside steel buckets within its diagnostic ward.\newline%
Rubin's technique avoided a need for a costly back{-}up in the hospital world.\newline%
The system requires just one match of a 150m rope core, which takes a full two hours to melt.\newline%
(Lieister Genre: Adventure, Photography)\newline%
Pneumophila causes a range of health and disease problems, including severe skin infections, epidemics, and brain diseases, according to the National Institute of Environmental Health Sciences.

%
\subsection{Image Analysis}%
\label{subsec:ImageAnalysis}%


\begin{figure}[h!]%
\centering%
\includegraphics[width=150px]{500_fake_images/samples_5_167.png}%
\caption{A Close Up Of A Black And White Striped Cat}%
\end{figure}

%
\end{document}