\documentclass{article}%
\usepackage[T1]{fontenc}%
\usepackage[utf8]{inputenc}%
\usepackage{lmodern}%
\usepackage{textcomp}%
\usepackage{lastpage}%
\usepackage{authblk}%
\usepackage{graphicx}%
%
\title{Differential chemokine expression in tubular cells in response to urinary proteins from patients with nephrotic syndrome}%
\author{Philip Gomez}%
\affil{Department of Pathology, Microbiology and Immunology, School of Medicine, University of South Carolina, Columbia, South Carolina, United States of America}%
\date{01{-}01{-}2013}%
%
\begin{document}%
\normalsize%
\maketitle%
\section{Abstract}%
\label{sec:Abstract}%
TUSTIN, Calif. {-} A cell therapy company called Neogenetics has reported that the fluoxetine treatment for stem cells achieved an immune response similar to that of chemotherapy in some patients who received it in the mid{-}1980s.\newline%
Scientists at the company began using mice with neural cell transplantation using human neural stem cells from the spinal cord and wrote that infection of the spinal cord was reduced up to 95 percent when neural stem cells were grown with phosphorylation of the anti{-}TNF chemotherapy drug.\newline%
Neogenetics President and CEO Mark W. Dilday said that most patients have success with other, more advanced stem cell treatments, but it is nearly impossible to treat all patients with HLA{-}A2 B cells transplanted outside the body. The treatment may become possible for patients by the end of this year, he said.\newline%
Despite a rapid shrinkage of disease, HLA{-}A2 B cells remain a persistent problem in transplanting the stem cells, Dilday said. But his company is working on a way to kill the cells and also create cells with the same DNA code as HLA{-}A2 B cells that have similar function, he said.\newline%
Neogenetics received \$50,000 from the Department of Energy's Cascade Catalyst Fund for its stem cell research.\newline%
Some of the same stem cells were also grown with different approaches after transplants to cancer patients. Two cells with different DNA and adeno{-}associated virus gene expression used for transplants in a woman with breast cancer were removed, isolated and grown again after others in the same patient were grown for stem cell transplantation in a bone marrow donor.\newline%
Dilday said that patients have used these new treatments on mice but that patients need more complex studies and may not get the same results with a different strategy in patients.\newline%
Several stem cell transplant sites have reported improved response to treatment with different dosing regimens. Dilday said the company has found those results come in relatively small numbers: one transplant site at the University of California, San Francisco, reported 23 percent reduction in disease growth after a three{-}week treatment period. At four more sites, 20 percent reduction in disease growth occurred in the first 10 days and 29 percent increase in disease growth over the four weeks followed by a 30 percent reduction.\newline%
Contact: Maria Cooper\newline%
(858) 751{-}1701

%
\subsection{Image Analysis}%
\label{subsec:ImageAnalysis}%


\begin{figure}[h!]%
\centering%
\includegraphics[width=150px]{500_fake_images/samples_5_181.png}%
\caption{A Man With A Beard Wearing A Tie And Glasses}%
\end{figure}

%
\end{document}