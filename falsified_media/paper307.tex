\documentclass{article}%
\usepackage[T1]{fontenc}%
\usepackage[utf8]{inputenc}%
\usepackage{lmodern}%
\usepackage{textcomp}%
\usepackage{lastpage}%
\usepackage{authblk}%
\usepackage{graphicx}%
%
\title{HIV{-}1 Tat Protein Increases Microglial Outward K+ Current and Resultant Neurotoxic Activity}%
\author{Austin Price}%
\affil{Division of Infection and Immunity, University College London, London, United Kingdom}%
\date{01{-}01{-}2013}%
%
\begin{document}%
\normalsize%
\maketitle%
\section{Abstract}%
\label{sec:Abstract}%
(November 21, 2012)  The National Cancer Institute (NCI) recently announced that Nrf2{-}Mediated Upregulation of Heme Oxygenase{-}1 in Human Oral Cancer Cells (HEE{-}Mediated) has been reported from a small group of cancer patients (96), as well as from other cancer cells that are exposed to very high doses of high amino acid compounds such as carbonate. HEE{-}Mediated results support a hypothesis that HEE{-}Mediated improves the expression of NCI{-}classified programs of the MAP kinase family, such as the Cancer B modulator identified by its role in nerve cell growth during growth{-}resistance events in those cells. Researchers reported oral findings on subjects with some form of APOLLO{-}related oral epithelial cells as well as cancer cells in an earlier clinical study. The oral results were presented at the 25th National Meeting \& Exposition of the American Society of Clinical Oncology.\newline%
Nrf2{-}Mediated could have significant implications for patient populations with several forms of cancers, including: (i) Oral cancers, which are generally harder to treat than others because of their susceptibility to apoptosis induced by high{-}active chemotherapy agents; (ii) B{-}cell and stem cell cancers, which are harder to treat because they typically are often treated with tyrosine kinase{-}inhibiting drugs; and (iii) patients with less common forms of cancer. Increased HEE{-}Mediated activity across this population of patients could mean a more complete understanding of the role Nrf2{-}Mediated plays in the various pathways metastasis to the gastrointestinal (GI) tract as well as the cell cycle in cancer cells. Researchers note that cancer cells may react differently to HEE{-}Mediated pharmacodynamic changes when their own factors are present, such as tumor suppressor proteins.\newline%
According to the NCI, researchers have shown that Nrf2{-}Mediated activity can be enhanced in as little as 5 micrograms of lung{-}derived oral beta{-}methyltransferase 9 (BMT{-}9), which is produced by epithelial cells in the colon, stomach, and spine. Scientists may be able to identify more effective drugs based on their impact on animal models, other indications, or cell{-}cycle{-}related outcomes.\newline%
Founded in 1986, Nrf2 is a naturally occurring and validated fundamental protein that regulates the regulation of physiological responses to oxidative stress, including apoptosis. Nrf2 research advances the understanding of the molecular mechanisms of cancer and mania. With focus on four key functions, Nrf2 technology enables researchers to uncover molecular mechanisms underlying cancer cell behavior and the genetic causes of human tumors. Learn more at: http://nrf2.com.

%
\subsection{Image Analysis}%
\label{subsec:ImageAnalysis}%


\begin{figure}[h!]%
\centering%
\includegraphics[width=150px]{500_fake_images/samples_5_376.png}%
\caption{A Man In A Suit And Tie Is Smiling}%
\end{figure}

%
\end{document}