\documentclass{article}%
\usepackage[T1]{fontenc}%
\usepackage[utf8]{inputenc}%
\usepackage{lmodern}%
\usepackage{textcomp}%
\usepackage{lastpage}%
\usepackage{authblk}%
\usepackage{graphicx}%
%
\title{DUSP1 Is a Novel Target for Enhancing Pancreatic Cancer Cell Sensitivity to Gemcitabine}%
\author{Elizabeth Yates}%
\affil{Department of Genetics, Washington University School of Medicine, St. Louis, Missouri, United States of America}%
\date{01{-}01{-}2008}%
%
\begin{document}%
\normalsize%
\maketitle%
\section{Abstract}%
\label{sec:Abstract}%
(Diploid peas)\newline%
The effects of ethanol in a plant plant are being investigated to determine whether they can be used in stem cell therapy for vitiligo, an inflammatory dermatologic disease of the skin and eye. The potential of this therapy is greatly limited because the resistance of this saponin is very high. However, some scientists do believe that the ethanol may have a healing effect. There are also current concerns about the resistance of the saponin. It is a natural toxin that causes cancer.\newline%
The problem is that while some cell pathways can be stimulated by ethanol alone, there is a large presence of the toxic element metabolite urea (U) in the cellular membranes of the plants. This is a concern because U is a precursor to U. All the natural extractions of U cause some mild allergic reactions (nosebleeds, stomachaches, vomiting and nausea) as well as any inflammation and cell damage. While alcohol and ethanol reduce U exposure, ethanol alone is still present.\newline%
So there are a number of possible anti{-}inflammatory effects of ethanol, which would include the elimination of the toxins that U can cause. One positive thing about ethanol is that it has limited amounts of toxic substances so that they will do not live on the outside. This means that the new therapies that may be developed after the research are aware of the potential ways in which U could be destroyed. With these limitations, it is not likely that any organisms would develop resistance to this toxin once taken orally.\newline%
If you have any questions regarding this or any field, please contact Elizabeth Brooker of Imperial College and on Twitter @IAmImperial.\newline%
We offer this written permission so the reader can come forward and share with you any facts and information. If you have information that the reader would appreciate, contact Imperial College Press Office at Imperial@imperial.ac.uk or tel. +44(0)1440 9463 664. You may fax and email your information to 020 7215 1659 or e{-}mail (press@iacpress.ac.uk).

%
\subsection{Image Analysis}%
\label{subsec:ImageAnalysis}%


\begin{figure}[h!]%
\centering%
\includegraphics[width=150px]{500_fake_images/samples_5_242.png}%
\caption{A Close Up Of A Zebra In A Window}%
\end{figure}

%
\end{document}