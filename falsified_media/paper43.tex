\documentclass{article}%
\usepackage[T1]{fontenc}%
\usepackage[utf8]{inputenc}%
\usepackage{lmodern}%
\usepackage{textcomp}%
\usepackage{lastpage}%
\usepackage{authblk}%
\usepackage{graphicx}%
%
\title{Genetic and epigenetic alterations are involved in the regulation of TPM1 in cholangiocarcinoma}%
\author{Brenda Herrera}%
\affil{Anesthesia Department, the Second Affiliated Hospital, School of Medicine, Xi'an Jiaotong University, Xi'an, P. R. China}%
\date{01{-}01{-}2014}%
%
\begin{document}%
\normalsize%
\maketitle%
\section{Abstract}%
\label{sec:Abstract}%
Reproductive laws may change if new draft regulations are adopted by Californias Board of Equalization. They can alter rules governing Californias Reproductive Freedom Act of 2007 and embryo taking and commercialization. The regulations passed by the Board of Equalization in 2007 are titled ProPO{-}System and Role of Phene{-}1{-}Like Regulation of the ProPO{-}System and Role of Phene{-}1{-}Like Cleaved Peptides in Innate Immunity.\newline%
The draft regulations provide for the exclusive assignment of the circumstances under which someone seeking medical treatment can leave a California hospital to travel or transfer a live pre{-}viability embryo to another facility and leave the delivery unit intact.\newline%
Representative Ben Kolker, R{-}Balboa, wrote in a letter to the Board of Equalization on December 13 to formally ask it to not approve the regulations. The resolution says that Governor Browns approval would subvert state human rights statutes. The Board of Equalization, in July 2009, refused a requested application for pregnancy and guidance to lower costs and lengthen the timeframe to implant. Other legislative efforts by the Board of Equalization failed in the fall of 2011.\newline%
Reproductive right advocates, most notably Senator Dianne Feinstein, in 2008 pushed to change the states law to allow termination at the point of abortion after the state determines that a woman may not provide care for a fetal or pre{-}viability organ. With that part of the California statute in mind, proponents considered whether a pregnancy could be terminated at a time when the woman could produce proof that the pregnancy did not have to be terminated. Phene{-}1{-}like regulation of the only conceivable link between exposure to certain products and causing induced abortion (a continuum of embryogenesis) is what they say distinguishes the current legislation from the past.\newline%
Feinstein is a staunch supporter of doctor{-}assisted abortions.\newline%
A case law archive in the National Institute of Justice related to the forcible terminations of human embryos is available online at: http://www.niknik.org/press/2012/09/30/64279.html\#ref{-}200484877

%
\subsection{Image Analysis}%
\label{subsec:ImageAnalysis}%


\begin{figure}[h!]%
\centering%
\includegraphics[width=150px]{500_fake_images/samples_5_138.png}%
\caption{A Black And White Photo Of A Bathroom Sink}%
\end{figure}

%
\end{document}