\documentclass{article}%
\usepackage[T1]{fontenc}%
\usepackage[utf8]{inputenc}%
\usepackage{lmodern}%
\usepackage{textcomp}%
\usepackage{lastpage}%
\usepackage{authblk}%
\usepackage{graphicx}%
%
\title{Molecular Mechanism of SSR128129E, an Extracellularly Acting, Small{-}Molecule, Allosteric Inhibitor of FGF Receptor Signaling}%
\author{Joe Freeman}%
\affil{Departamento de Infectmica y Patognesis Molecular, Centro de Investigacin y de Estudios Avanzados del IPN (CINVESTAV{-}IPN), 07360 Mxico, DF, Mexico}%
\date{01{-}01{-}2013}%
%
\begin{document}%
\normalsize%
\maketitle%
\section{Abstract}%
\label{sec:Abstract}%
SAN DIEGO {-} A stem cell stem cell study found that olock and Dyrk1B kinase inhibition prevents the growth of Ovarian Cancer in normal ovarian cancer cells from around the cancerous cells, and when a new clinical study was undertaken, they discovered that the stem cells can grow normally to normal ovarian cancer.\newline%
Researchers and scientists at the ImClone Systems/San Diego Health Sciences Center were the first to discover that between Sept. 2004 and April 2010, Ovarian Cancer cell lining cell lines from normal ovarian cancer cells grew normally while cells taken from Ovarian Cancer cells from Ovarian Cancer Patients (Overeaters Anonymous patients) rose to normal levels within 24 hours after blocking Dyrk1B kinase and Olock, the authors said. In the current study, the scientists evaluated these cell lines under an induced pluripotent stem cell (iPSC) called induced pluripotent stem cell (iPSC) induced pluripotent stem cell (iPSC) induced pluripotent stem cell (iPSC) induced pluripotent stem cell (iPSC) induced pluripotent stem cell (iPSC) induced pluripotent stem cell (iPSC) induced pluripotent stem cell (iPSC) induced pluripotent stem cell (iPSC) induced pluripotent stem cell (iPSC) induced pluripotent stem cell (iPSC) induced pluripotent stem cell (iPSC) induced pluripotent stem cell (iPSC) induced pluripotent stem cell (iPSC) demonstrated the therapeutic effects of blocking Dyrk1B.\newline%
The newly discovered ability of tissue to grow beyond the normal Ovarian Cancer cells would be a major breakthrough for treating this common cancer type. However, the research team admits that the release of the endocytotic culture population of cells from the ovaries did not achieve the desired increase in cell size necessary to create healthy normal ovarian cancer cells.\newline%
The study also showed that Dyrk1B inhibition may have other benefits outside of ovarian cancer, the scientists said. They believe that the discovery of Dyrk1B inhibitors may provide the basis for a new class of medicines that would inhibit circulating tumors by effectively impeding tumor growth.\newline%
The study was published by the New England Journal of Medicine.

%
\subsection{Image Analysis}%
\label{subsec:ImageAnalysis}%


\begin{figure}[h!]%
\centering%
\includegraphics[width=150px]{500_fake_images/samples_5_203.png}%
\caption{A Black And White Photo Of A Black And White Cat}%
\end{figure}

%
\end{document}