\documentclass{article}%
\usepackage[T1]{fontenc}%
\usepackage[utf8]{inputenc}%
\usepackage{lmodern}%
\usepackage{textcomp}%
\usepackage{lastpage}%
\usepackage{authblk}%
\usepackage{graphicx}%
%
\title{TNF{-}a down{-}regulates the NaCeKC ATPase and the Na+{-}K+{-}2Cl{-} cotransporter in the rat colon via PGE2}%
\author{Patrick Lewis}%
\affil{Zhang Zhongjing College of Chinese Medicine, Nanyang Institute of Technology, China}%
\date{01{-}01{-}2014}%
%
\begin{document}%
\normalsize%
\maketitle%
\section{Abstract}%
\label{sec:Abstract}%
Researchers at UC San Diego have found that SANS{-}1 creates a lasting effect on metabolization of the pathogen during infection, leading to the excessive production of prooxymidine. The findings suggest that SANS{-}1 may hinder the effectiveness of pharmaceutical therapies for Influenza Virus infection (I3A).\newline%
This finding has important implications for the management of influenza in North America. The WHO estimates that less than 15\% of the population respond positively to the first doses of an influenza vaccine given, and that the lower proportion is causing clinical complications as well as disease progression. Such complications could lead to substantial financial losses as well as public health, public{-}policy and medical{-}society effects in many countries where vaccination is widespread.\newline%
In an especially large study of Influenza (I3A), the authors defined the dose point at which the protein production in SANS{-}1 or its metabolized form is normal. Next, they tracked the activity of antibody neutralization proteins (ADSP) during various doses of I3A. As instructed by the protocol, the researchers sent positive human antibody to arms of five influenza virus strains, matched to clinical samples from a range of volunteers. Their patients showed normal levels of anti{-}virals antibodies but also evidence of a loss of PNP enzyme function which led to an increased production of prooxymidine.\newline%
Given these findings, Dr. Robert Crimatto and members of the Karolinska Institute team will be conducting a clinical study in which they will see the best defense against Influenza is to prevent production of SANS{-}1, even at doses in the approved range. Their role in modifying the immune response against SANS{-}1 in people has implications for surveillance and preventive surveillance in conjunction with higher doses of antibody vaccination.\newline%
This paper was previously published online in the Journal of the Royal Society of Chemistry in June 2013. A commentary by C.P. Brusberg, Peter Toner, and coauthors stated: The magnitude of adverse events arising from an increased elevation in prooxymidine levels that are especially important for the treatment of influenza primarily within a patient population with lymphohapide{-}nave clinical experience is unique.

%
\subsection{Image Analysis}%
\label{subsec:ImageAnalysis}%


\begin{figure}[h!]%
\centering%
\includegraphics[width=150px]{500_fake_images/samples_5_329.png}%
\caption{A Man In A Tie Is Standing In A Room}%
\end{figure}

%
\end{document}