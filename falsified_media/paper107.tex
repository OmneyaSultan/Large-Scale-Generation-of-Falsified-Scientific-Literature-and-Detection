\documentclass{article}%
\usepackage[T1]{fontenc}%
\usepackage[utf8]{inputenc}%
\usepackage{lmodern}%
\usepackage{textcomp}%
\usepackage{lastpage}%
\usepackage{authblk}%
\usepackage{graphicx}%
%
\title{JMJD6 is a driver of cellular proliferation and motility and a marker of poor prognosis in breast cancer}%
\author{Amy Wu}%
\affil{National Key Laboratory for Crop Genetics and Germplasm Enhancement, Jiangsu Plant Gene Engineering Research Center, Nanjing Agricultural University, Nanjing, 210095, China}%
\date{01{-}01{-}2013}%
%
\begin{document}%
\normalsize%
\maketitle%
\section{Abstract}%
\label{sec:Abstract}%
We hear about immunosuppressive agents to prevent cancer from ever taking hold, but those approaches are not workable in earlier stages of disease. FCT is the most aggressive, under{-}served and difficult to treat form of lymphoblastic leukemia. If its cell death rate could be decreased enough, this patient might be able to prolong his or her life.\newline%
Recently, Dr. Jim Phillips of the University of Miami and colleagues at the National Institutes of Health in Maryland have shown how BKM120 is involved in B{-}chronic lymphocytic leukemia, an aggressive form of leukemias that can be spread through blood transfusions.\newline%
The technique the team used was using novel electrophysiology techniques to dramatically increase the concentration of the alpha isoform in the B{-}chronic lymphocytic leukemia cells used as biochemical models.\newline%
When they were first first informed that certain cancer cells, including cancer cells that express a protein called alpha{-}angiocortin{-}2 (AN2), were the most frequently expressed alpha isoform, they began to advocate a treatment strategy that might stop and weaken the properties of BKM120, a potent PA kinase inhibitor, which is used in some previously unheard of cancers.\newline%
Advertisement\newline%
These preclinical studies have provided them the potential to translate their argument into treatment for patients with B{-}chronic lymphocytic leukemia.\newline%
BKM120 may be active in brain diseases such as Alzheimers and Parkinsons, yet it is less well studied in B{-}chronic lymphocytic leukemia. It may also be active in other chronic and acute forms of the disease.\newline%
It is a versatile drug that may be able to treat these forms of cancer, as well as other diseases in which BKM120 may play a role. This drug, if effective, may have tremendous potential to save lives in this disease.\newline%
Note: The author is Contributing Scientist with Richard Chutkan{-}Gajda of the National Institute of Neurological Disorders and Stroke, so your reading may include NINDS publication links.\newline%
AlertMe

%
\subsection{Image Analysis}%
\label{subsec:ImageAnalysis}%


\begin{figure}[h!]%
\centering%
\includegraphics[width=150px]{500_fake_images/samples_5_196.png}%
\caption{A Man Taking A Picture Of Himself In A Mirror}%
\end{figure}

%
\end{document}