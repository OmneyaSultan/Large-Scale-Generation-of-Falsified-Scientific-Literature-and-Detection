\documentclass{article}%
\usepackage[T1]{fontenc}%
\usepackage[utf8]{inputenc}%
\usepackage{lmodern}%
\usepackage{textcomp}%
\usepackage{lastpage}%
\usepackage{authblk}%
\usepackage{graphicx}%
%
\title{The Helicobacter pylori Urease B Subunit Binds to CD74 on Gastric Epithelial Cells and Induces NF{-}\_\_B Activation and Interleukin{-}8 Production}%
\author{Paul Carter}%
\affil{Bellvitge Biomedical Research Institute (IDIBELL), Barcelona, Spain}%
\date{01{-}01{-}2013}%
%
\begin{document}%
\normalsize%
\maketitle%
\section{Abstract}%
\label{sec:Abstract}%
Researchers from UC San Diego discovered a new drug that inhibits proteasome inhibition {-}{-} a group of enzymes that control gene expression {-}{-} by inhibiting proteasome inhibition, also known as proteasome expression inhibition. The findings were published in Proceedings of the National Academy of Sciences.\newline%
I chose the drug I called PROPEL6 to better investigate the role of proteasome inhibition in leukemia cells. The National Cancer Institute sponsored a study at UCSD, published in the November 8 issue of Cell Metabolism, in which 48 ovarian cancer cell lines were implanted in controlled cell cultures. The polygenic mice were genetically engineered to harbor the early stage I leukemia cancer cells.\newline%
After several years of previous research, pharmacologists Yegen Sun and Tam Fatih, both of UC San Diego, identified that PROPEL6 is induced by proteasome inhibition in these early{-}stage leukemia cells. After further experimentation, these inhibitors were found to induce apoptosis in malignant leukemia cells.\newline%
After the colonoscopy, the cells were kept under strict daily monitoring for apoptosis during observation. PROPEL6 was found to cause apoptosis in two year old mice. Also, these individuals who received the drug had heightened notches on their endocrines, tissues that are responsible for the spread of disease.\newline%
"Conventional therapies for ovarian cancer depend on cytoplasmic hyperstimulation (Cytoplasmic Stimulation) and anti{-}Lympho{-}Mason (Ammono{-}Mason.3) to transform ovarian cancer cells," said Wang. "The non{-}cytoplasmic strategy does not produce long{-}term protective effects; however, both pharmacologic and proteasome inhibition are efficacious in curbing ovarian cancer tumor growth."\newline%
PROPEL6 inhibits to a lesser extent proteasome inhibition in ovarian cancer cells harboring B{-}cell lymphocytes and Beta{-}Mason. Anti{-}Lympho{-}Mason is a member of the families of chromatophores that produces half of all the B cells in the human body.\newline%
PROPEL6 also effects cells that form the bottom of the B cell wall, which is responsible for the B cell's decision{-}making. Patients with cancer have lower levels of B cell as well as cytotoxic T cells. Nanotechnology research may lead to a far lower burden of chromatophores for ovarian cancer, said the scientists.\newline%
Source:\newline%
Heather Regan{-}Schwartz\newline%
UC San Diego

%
\subsection{Image Analysis}%
\label{subsec:ImageAnalysis}%


\begin{figure}[h!]%
\centering%
\includegraphics[width=150px]{500_fake_images/samples_5_200.png}%
\caption{A Close Up Of A Person Holding A Tooth Brush}%
\end{figure}

%
\end{document}