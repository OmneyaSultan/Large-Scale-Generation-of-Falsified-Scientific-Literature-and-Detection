\documentclass{article}%
\usepackage[T1]{fontenc}%
\usepackage[utf8]{inputenc}%
\usepackage{lmodern}%
\usepackage{textcomp}%
\usepackage{lastpage}%
\usepackage{authblk}%
\usepackage{graphicx}%
%
\title{Neurogenesis and Increase in Differentiated Neural Cell Survival via Phosphorylation of Akt1 after Fluoxetine Treatment of Stem Cells}%
\author{Russell May}%
\affil{Blood Transfusion Centre of Slovenia, Ljubljana, Slovenia}%
\date{01{-}01{-}2013}%
%
\begin{document}%
\normalsize%
\maketitle%
\section{Abstract}%
\label{sec:Abstract}%
in combination with zoledronic acid: A synergistic effect in triple{-}negative breast cancer cell lines\newline%
It seems oxytetracycline (TM) and verinolone acetate (VAA) and Zoledronic acid (ZO) are the prescription drug combination most commonly used for combination treatment of triple{-}negative breast cancer cell lines. However, most patients who are willing to give this particular combination a try are doing so because it has been shown to improve overall survival in triple{-}negative breast cancer patients.\newline%
K{-}Ac flat fluorescent fluorescent markers (LFD) are used as colors to indicate when a specific substance is present. The technology in place allows an insertion of a magnet to pick up on the slightest changes in the light. In a paper that was recently published in the journal Proceedings of the National Academy of Sciences (PNAS), data from a multisite, randomized clinical trial of triple{-}negative breast cancer cell lines (mutant triple{-}negative cells genetically altered to increase the effect of transthyretin (TTR) on tumor growth.\newline%
The results shown by the investigators suggest that in some cases, the combination of Zoledronic acid and prefetimer (LFN) is as safe and effective as azathioprine and oxytetracycline, respectively. Mutated triple{-}negative breast cancer cell lines were not tested in any triple{-}negative breast cancer cell lines that were not genetically modified to grow TTR. For the predominantly FDA{-}approved transthyretin inhibitor, the combination proved to be as safe and effective as cisplatin and Zoledronic acid.\newline%
Zoledronic acid is already approved for the treatment of certain laboratory disorders, but with its convenience in dosage, the data suggest that the PNP{-}190 study showed that at 3 months, hemolytic uremic syndrome (HUS) and cystic fibrosis only occur for 29 percent of patients treated with zoledronic acid and cisplatin, respectively. This means that zoledronic acid could provide long{-}term protection against HUS or cystic fibrosis mutations.\newline%
Written by Holly Kaye{-}Newton

%
\subsection{Image Analysis}%
\label{subsec:ImageAnalysis}%


\begin{figure}[h!]%
\centering%
\includegraphics[width=150px]{500_fake_images/samples_5_472.png}%
\caption{A Close Up Of A Person Wearing A Suit And Tie}%
\end{figure}

%
\end{document}