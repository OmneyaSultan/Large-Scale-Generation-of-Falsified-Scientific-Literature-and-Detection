\documentclass{article}%
\usepackage[T1]{fontenc}%
\usepackage[utf8]{inputenc}%
\usepackage{lmodern}%
\usepackage{textcomp}%
\usepackage{lastpage}%
\usepackage{authblk}%
\usepackage{graphicx}%
%
\title{X{-}Box Binding Protein 1 (XBP1s) Is a Critical Determinant of Pseudomonas aeruginosa Homoserine Lactone{-}Mediated Apoptosis}%
\author{Daniel Graves}%
\affil{INSERM, U895 (quipe 1), Equipe lablise Ligue Contre le Cancer, C3M, 06204 Nice, France}%
\date{01{-}01{-}2003}%
%
\begin{document}%
\normalsize%
\maketitle%
\section{Abstract}%
\label{sec:Abstract}%
In this discovery, researchers identified a newly discovered group of small molecule antagonists designed to drastically reduce the quantity of caz and tom growth factors present in E. coli.\newline%
By studying this dDNA sequence, the team demonstrated that these small molecules play an important role in inflammation, cancer and even boosting demand for anti{-}HIV drugs such as Viivilcemes (Quimum) and Exforge. The results presented at the 13th Genome Showcase at the ESMO 2014 conference in Madrid, Spain.\newline%
E. coli is a serious agricultural bug, and the number of contaminated strains of E. coli has tripled over the past 25 years. Genetic defects reduce the number of normal strains, while they increase the number of genetically defective ones, which has led to the emergence of a related virus, Salmonella, caused by Streptococcus (rat) in eggs and with commercial genetic engineering.\newline%
Strep microbe{-}identification techniques have shown that the cause of outbreaks can be based on trace numbers from genetically different bacteria, and this technique has also shown that the trace numbers can be used to test for a host of other risk factors, which can have consequences for disease.\newline%
In their cell culture study, the researchers found that these three small molecules reduce circulating Caz growth factors by at least 90\% when studying the endogenoussomes of various strains of E. coli. Additionally, these molecules reduce the quantity of protozoa representing the proteases that take their cues from Caz growth factors. The cytosynchromes of E. coli cells contained a novel epigenetic code that took the pathologies from normal cells to Ha new form of Hand thereby reduced the production of the protozoa that eliminate Caz by at least 90\% and increased production of the protozoa that create Caz (harmful protozoa).

%
\subsection{Image Analysis}%
\label{subsec:ImageAnalysis}%


\begin{figure}[h!]%
\centering%
\includegraphics[width=150px]{500_fake_images/samples_5_172.png}%
\caption{A Man With A Beard Wearing A Tie}%
\end{figure}

%
\end{document}