\documentclass{article}%
\usepackage[T1]{fontenc}%
\usepackage[utf8]{inputenc}%
\usepackage{lmodern}%
\usepackage{textcomp}%
\usepackage{lastpage}%
\usepackage{authblk}%
\usepackage{graphicx}%
%
\title{X{-}Box Binding Protein 1 (XBP1s) Is a Critical Determinant of Pseudomonas aeruginosa Homoserine Lactone{-}Mediated Apoptosis}%
\author{Andrew Jones}%
\affil{Breast Disease Center, Southwest Hospital, Third Military Medical University, Chongqing, P.R. China, Department of Pathology, The Fourth Hospital of Hebei Medical University, Shijiazhuang, P.R. China, Department of Breast Disease Center, The Fourth Hospital of Hebei Medical University, Shijiazhuang, P.R. China}%
\date{01{-}01{-}2012}%
%
\begin{document}%
\normalsize%
\maketitle%
\section{Abstract}%
\label{sec:Abstract}%
Researchers from the Salk Institute for Biological Studies have discovered that inhibiting polymorphism in the matrix moron, a key juncture of the insulin pathway, can contribute to oral cancer cell metastasis. The findings suggest potential new strategies for further immunotherapy research.\newline%
One of the first studies exploring whether a line of myelin would make tumor cells inflexible and pose a significant barrier to anticancer therapies, the sarcomaoma cell metastasis mice in mice bred to have low myelin were found to have significantly less tumor cell proliferation (i.e., tumor cell multiply without myelin), a function critical to tumor development.\newline%
Lead author Anna Maria Rivas{-}Rubato, M.D., an investigator at the Salk Institute's Department of Pathology, DDS, chief of sarcoma research and chair of the agency's department of genetic medicine, suggests that Myelin inhibition of sarcoma cell migration to other, healthier locations in the tumor could be used to more effectively engage the targets that these mice were undergoing cancer treatment.\newline%
In 2009, Salk staff reported that these mice were suffering acute tumor cell death and the tumors had begun to spread, although exact cause of this switch in cell migration to other tumor sites had not been shown. In these study mice, where there were no residual Myelin asymptomatic tumors, tumor cell migration shifted from myelin to base or structure.\newline%
The mice were fed either standard chemotherapy or dietary medication containing sulforaphane to increase the myelin barrier called the relapsing portal, which prevents expansion of the tumor. These trials showed that daily diets of sulforaphane inhibited cancer cell migration to other locations in the tumor.\newline%
It has been shown in previous findings that sulforaphane decreases the rate of cancer cell proliferation, and in November 2010, Rivas{-}Rubato and colleagues from the University of Arizona's Institute for Clinical and Translational Science disclosed that sulforaphane may also inhibit tumor cell metastasis. This discovery was published in the journal Cancer.\newline%
"In previous studies, prior to release of this study, the hypothesis that sulforaphane would inhibit metastasis was considered to be flawed," said Rivas{-}Rubato. "The latest results will provide us with a new tool to fight cancer in mice. We now have better control over molecular pathways and the ability to develop treatments for astrocytoma. A few years ago, this research would have been too hard to do. Now, by using telomerase inhibitors (truzyme inhibitors) we can accelerate chemotherapy in mouse models."\newline%
Rivas{-}Rubato and colleagues are planning to continue their clinical study with the sarcoma mice in order to determine the potential side effects of daily abuse of sulforaphane and utilize functional MRI findings to gain better insight into the effectiveness of levofloxacin, the approved drug for gastric cancer, as a treatment for sarcoma.

%
\subsection{Image Analysis}%
\label{subsec:ImageAnalysis}%


\begin{figure}[h!]%
\centering%
\includegraphics[width=150px]{500_fake_images/samples_5_383.png}%
\caption{A Man In A Suit And Tie Holding A Teddy Bear}%
\end{figure}

%
\end{document}