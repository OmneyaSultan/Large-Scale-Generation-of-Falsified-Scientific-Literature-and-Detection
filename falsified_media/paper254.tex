\documentclass{article}%
\usepackage[T1]{fontenc}%
\usepackage[utf8]{inputenc}%
\usepackage{lmodern}%
\usepackage{textcomp}%
\usepackage{lastpage}%
\usepackage{authblk}%
\usepackage{graphicx}%
%
\title{Terminalia catappa Exerts Antimetastatic Effects on Hepatocellular Carcinoma through Transcriptional Inhibition of Matrix Metalloproteinase{-}9 by Modulating NF{-}\_B and AP{-}1 Activity}%
\author{Kimberly Romero}%
\affil{Department of Radiation Medicine, Institute of Modern physics, Chinese Academy of Sciences, Lanzhou, China, \newline%
    Key Laboratory of Heavy Ion Radiation Biology and Medicine of Chinese Academy of Sciences, Lanzhou, China, \newline%
    Key Laboratory of Heavy Ion Radiation Medicine of Gansu Province, Lanzhou, China}%
\date{01{-}01{-}2013}%
%
\begin{document}%
\normalsize%
\maketitle%
\section{Abstract}%
\label{sec:Abstract}%
New research at ENCORE (Easton Physiology, Oregon Research Institution), located at the University of California, Santa Barbara, shows a different mechanism that must be involved with a level of somnolence activation, known as a tounilius, that occurs in the innate immune system to trigger a TFRi6 receptor called Gram{-}1, T{-}helix function, or GZKK, tolerance. The role of Gram{-}1 in the transition from T{-}helix to TFRi6 modulation (also known as Archem/T{-}T{-}Co) is central to the understanding of why sometimes human cells colonize and reproduce in a genetically selected, genetically defined pathway, and in the careful morphological migration of genes encoding antibody complexes that are desired for role in global and biomagnified immunity. The new study is available online in the early stages of publication.\newline%
Scientists have long known that the most common colonizing bacteria  the most common member of the gut microbiome known to enter the gut, including many beneficial microbes  is harmless. Instead, their task is to colonize our intestinal wall, in order to provide our bodies with all the food they need. It is this kind of rapid and surgical receipt of microbes that many feared would be disastrous for our immune system and weeded out the fungal invaders. But the typical approach is to use antibiotics to kill all the wonderful microbes in our guts as quickly as possible.\newline%
The new ENCORE study is the first to directly correlate these slow and controlled breeding pathways by gut{-}microbiologists with how immunity is transmitted in humans. The researchers identified a special route for the efficient transfer of genes between these two lines of genetic communication. They sequenced the system and found that one of the links, known as the tounilius, is primarily involved in the transfer of genes from T{-}helix to TFRi6. The team identified two additional pathways. Those are outside the tounilius, but in close proximity to it in the neighboring bacterial population (JcPS2). In other words, they use it as a relay for the transfers from T{-}helix to TFRi6.\newline%
Scientists have known for decades that this mechanism need not be constantly activated in the presence of bacteria. Tounilius produces somnolence, a molecule that decreases contact between germ cells and outer walls of the immune system. However, this interaction with bacterial roots may promote some harmful bacterial enter the gut and can weaken immunity to many strains of common bacterial and viral pathogens.\newline%
The team discovered a way for T{-}helix to convert energy needed for their absorption from bacterial species into TFRi6 to slow resistance in the gut. Specifically, by using this avoidance mechanism the scientists are able to reverse the resulting TFRi6 conversion to the receptors that are critical for the cascade of cellular activity that creates immunity.\newline%
The new study provides a small first step toward achieving the long{-}term goal of suppressing the excess responses to these gut{-}microbiological changes that are needed to prevent uncontrolled intestinal colonization. The study could also provide a new molecular model for the restoration of the normal bacterial population, further elucidating the developmental path through which antibiotic resistance begins.\newline%
The work was supported by grants from the National Institute of Allergy and Infectious Diseases at the National Institutes of Health, and the U.S. Department of Health and Human Services.

%
\subsection{Image Analysis}%
\label{subsec:ImageAnalysis}%


\begin{figure}[h!]%
\centering%
\includegraphics[width=150px]{500_fake_images/samples_5_328.png}%
\caption{A Man In A Suit And Tie Is Holding A Camera}%
\end{figure}

%
\end{document}