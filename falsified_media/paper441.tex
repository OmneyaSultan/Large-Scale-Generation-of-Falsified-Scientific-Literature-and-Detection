\documentclass{article}%
\usepackage[T1]{fontenc}%
\usepackage[utf8]{inputenc}%
\usepackage{lmodern}%
\usepackage{textcomp}%
\usepackage{lastpage}%
\usepackage{authblk}%
\usepackage{graphicx}%
%
\title{Baicalein Reduces the Invasion of Glioma Cells via Reducing the Activity of p38 Signaling Pathway}%
\author{Jorge Rogers}%
\affil{Zhang Zhongjing College of Chinese Medicine, Nanyang Institute of Technology, China}%
\date{01{-}01{-}2014}%
%
\begin{document}%
\normalsize%
\maketitle%
\section{Abstract}%
\label{sec:Abstract}%
(San Diego) {-} Scientists at the National Cancer Institute (NCI) and National Institutes of Health (NIH) have found a previously unappreciated genetic mutation that increases expression of the C{-}reactive protein gene (CREBH) in the embryonic stem cells (ESCs) of experimental cancer cells. CRISPR/Cas9 (CRISPR/Cas9) advances its understanding of gene expression in human cell lines and human tissues and prepares us for a future of genome editing.\newline%
Cases of xenograft (editing or inserting foreign DNA) used for clinical purposes involve damaging cells such as prostate cancer and heart disease by inserting foreign DNA strands into human cells. These tools have posed unique challenges in identifying and manipulating the specific genes responsible for these cancer cell lines.\newline%
Genetic discoveries indicated that a type of mutation in the CREBH gene significantly increased gene expression in human ESCs for colon cancer, HER2 breast cancer, EGFR oncogene in breast cancer, and early{-}stage lung cancer. A study published in the journal Cell Reports indicated that this pathway also results in higher rate of chromosomal transfer and replacement, and that genetic alterations in this pathway that result in the transfer of chromosomal DNA into cells of non{-}cancerous animals may contribute to genetic traits that may lead to mortality. CRISPR/Cas9 is the most widely used gene editing tool in human biology, but is not used in cell lines of cell{-}based pluripotent stem cells (PSCs). The process of cell{-}based pluripotent stem cell stem cell proliferation has caused some researchers to question its use in human cells.\newline%
Researchers at the NIH determined that the C{-}reactive protein gene that CREBH regulates, C{-}reactive protein (C{-}RPR), is expressed in embryonic ESCs using a study that identifies the tumor suppressor gene{-}producing chimeric antigen receptor (CAR) interleukin. Researchers followed various ESCs to detect that CREBH had increased expression in ESCs. A team of scientists demonstrated that by activating the C{-}RPR in ESCs with the same mutation associated with CREBH, CREBH was stimulated to generate activation of cell{-}like ligand (CL) receptor 2 kinase (CLK2). Based on these results, researchers showed that the C{-}RPR gene in cell{-}based cells is manipulated to activate the C{-}RPR gene in a variety of cells, including ESCs derived from mouse embryonic stem cells. The team also showed that CREBH was also activated when C{-}RPR was manipulated in human ESCs derived from cells derived from mouse models of HER2 and HER2 positive breast cancer and EGFR oncogene.\newline%
The study was funded by NCI to pursue the application of CRISPR/Cas9 to normalize gene expression in ESCs from CAR cells.

%
\subsection{Image Analysis}%
\label{subsec:ImageAnalysis}%


\begin{figure}[h!]%
\centering%
\includegraphics[width=150px]{500_fake_images/samples_5_497.png}%
\caption{A Close Up Of A Small Black And White Cat}%
\end{figure}

%
\end{document}