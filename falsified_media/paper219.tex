\documentclass{article}%
\usepackage[T1]{fontenc}%
\usepackage[utf8]{inputenc}%
\usepackage{lmodern}%
\usepackage{textcomp}%
\usepackage{lastpage}%
\usepackage{authblk}%
\usepackage{graphicx}%
%
\title{Electroacupuncture Treatment Improves Neurological Function Associated with Regulation of Tight Junction Proteins in Rats with Cerebral Ischemia Reperfusion Injury}%
\author{Mary Harrington}%
\affil{Department of Surgery, Gastroenterological Surgery, Graduate School of Medicine, Osaka University, Suita, Osaka, Japan}%
\date{01{-}01{-}2013}%
%
\begin{document}%
\normalsize%
\maketitle%
\section{Abstract}%
\label{sec:Abstract}%
The name of Shigella flexneri, once another fecal matter bacteria known as octopus and find. Up until 2011 researchers at California State University San Marcos did not know that Shigella f{-}F would also share the same name with cancer progesterone, a drug used in some surgeries for a tumor on the cervix. Researchers at UC San Diego also began seeing similarities of Shigella flexneri in the urine of patients who had previously received chemotherapy.\newline%
In October 2011 UC San Diego immunologist Chiapas Arevalo and colleague Cheryl Lee launched the Shigella Freebridge project, a collaborative effort with UC San Diego health center researchers to further study the relationship between Injection Drug Incense and cancer properties in animals. Initially funded through a grant from the UC San Diego Endowment, Injection Drug Incense is a class of drugs that contains Atripla and propionate dehydrogenase 2 (ADH2) inhibitors. As a first step it was suggested that the arsenic nucleotides in Injection Drug Incense could be used as the aptamers to bind to or inhibit cancer progesterone. The researchers set out to understand whether Injection Drug Incense could bind together the genome of an antigen and induce the cancer progesterone enzyme to turn on.\newline%
Intrigued, Arevalo and his colleagues started using an immunogenic environment to test Injection Drug Incense and carcinogenic stromal antigen. This experiment was so successful that the team later conducted two other experiments. Toxin levels in the bloodstream of a three year old brother and sister, both identical twins, were collected by genetic material they were given when they were infants. During a follow up visit they found traces of the cancer progesterone enzyme that had been present in their mothers' tissues.\newline%
They did the same with cancer progesterone in postmenopausal women. The two studies supported the hypothesis that increasing levels of the cancer progesterone enzyme would alter the growth pattern of Chagas (Methosceleta prion) within the body's colon. An infection of the colon from chronic colonitis was detected in 39\% of the mothers of American Children's Hospital transplant recipients and in 9\% of the mothers of Western Connecticut Hospital transplant recipients. The results were consistent with the radiographic findings of other studies.\newline%
The team then studied Injection Drug Incense from 2009 through 2010 to understand how the ability of cancer progesterone to rise to the surface of the colon translated to cancer growth in the colon, tumor cells and colon organ walls. They encountered resistance to the chemotherapy drugs at lower concentrations and identified the receptor protein interleukin{-}20 (IL{-}20) as the neural pathway responsible for the pathway diverging to cancer growth, the brain, liver and pancreatic atrium. They found that the pathway responds favorably to chemotherapy therapy and that was the focus of the TACART{-}12 study.\newline%
Additionally, researchers have identified the neurotransmitter glutamate as a target for cancer progesterone. In the past three years researchers have performed experiments on mice that showed that the protein glutamate over{-}represents the tumor cells and in order to induce tumor growth uptake, glutamate was suspended in the mice. Their targets for the proliferation of cancer cells can be identified by blocking glutamate uptake.\newline%
These results finally bring a link to cancer progesterone, a drug used for this purpose.\newline%
It is believed that the sensitivity of tumor cells to cancer progesterone is based on its interaction with the drug Estradiolin, an amino acid used to bind cancer progesterone. At present there is no evidence that Estradiolin interaction directly effects cancer growth; in principle, however, it may explain the increase in cancer growth rates observed in the published studies.\newline%
In an effort to advance the understanding of the gene of cancer progesterone, and potentially to predict which genes are associated with cancer progesterone, UCLA researchers studied a protein that was activated on tumor cells and found its expression in tumors substantially correlated with tumor growth.\newline%
The Shigella Freebridge project is now examining how the expression of Estradiolin

%
\subsection{Image Analysis}%
\label{subsec:ImageAnalysis}%


\begin{figure}[h!]%
\centering%
\includegraphics[width=150px]{500_fake_images/samples_5_297.png}%
\caption{A Man In A Suit And Tie Holding A Teddy Bear}%
\end{figure}

%
\end{document}