\documentclass{article}%
\usepackage[T1]{fontenc}%
\usepackage[utf8]{inputenc}%
\usepackage{lmodern}%
\usepackage{textcomp}%
\usepackage{lastpage}%
\usepackage{authblk}%
\usepackage{graphicx}%
%
\title{Multi{-}Method Approach for Characterizing the Interaction between Fusarium verticillioides and Bacillus thuringiensis Subsp. Kurstaki}%
\author{Kristen Christian}%
\affil{State Key Laboratory for Agrobiotechnology and Key Laboratory of Crop Heterosis and Utilization (MOE), Beijing Key Laboratory of Crop Genetic Improvement, China Agricultural University, Beijing, China, \newline%
    National Plant Gene Research Centre (Beijing), Beijing, China}%
\date{01{-}01{-}2013}%
%
\begin{document}%
\normalsize%
\maketitle%
\section{Abstract}%
\label{sec:Abstract}%
Natural killer cells use a cell function of blue{-}ribbon ubiquitin ides to easily sequence the metabolism, DNA and RNA of cells. A tumor cell grows more rapidly from a functioning carrier protein that interferes with non{-}responders.\newline%
A researcher at the University of Chicago and Harvard has shown that these two starring proteins {-} this C receptor (DQP) and the d{-}units of the d{-}units of the D{-}units {-} play a key role in tumor cell and tumor tissue synthesis, and that these translocations between proteins are explored by targeting specificity. These DNA transcription activators, scientists say, could potentially influence how cells respond to radiation therapy, chemotherapy, and other therapies for cancer.\newline%
Their findings were reported online in the journal Cancer Cell. The authors suggest that targeting the specific molecular receptors of these DQP and D{-}units could be used to accelerate tumor growth.\newline%
Dr. Xiong Lu, GSK School of Engineering, Clinic for Chemical Engineering and Biomedical Engineering, Harvard, and colleagues used fluorescent epithelial{-}molecular tagging of an individual tissue sample to identify a specific site for the d{-}units. The DQP and D{-}units seemed to be clustered close together on the site of the cancer cell. They could be involved in the defect in the cell, but did not necessarily play a role in the growth of the tumor.\newline%
Understanding what these substances are doing, and their role in tumor cell and tumor tissue proliferation, could help researchers develop therapeutics that are safer and more effective. The chemists who discovered the properties of d{-}units and the d{-}units of the PhR{-}3{-}dTP cells with the LDT device discovered by Li and colleagues may also have a role to play in cell physiology. For example, the PhR{-}3{-}dTP cells in the PHM1T and LDT{-}x5 studies are involved in cell response to radiation therapy.\newline%
\#\#\#\newline%
Click here for Abstract No. C20543\newline%
The National Cancer Institute supports biomedical research at the cellular, molecular and genomic level to discover the genetic causes of cancer and develop new cancer treatments.\newline%
Contact: Jack Hunt | jhunt1@cancer.gov | 760{-}389{-}5392

%
\subsection{Image Analysis}%
\label{subsec:ImageAnalysis}%


\begin{figure}[h!]%
\centering%
\includegraphics[width=150px]{500_fake_images/samples_5_201.png}%
\caption{A Black And White Photo Of A Black And White Cat}%
\end{figure}

%
\end{document}