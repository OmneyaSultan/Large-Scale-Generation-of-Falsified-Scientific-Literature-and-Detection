\documentclass{article}%
\usepackage[T1]{fontenc}%
\usepackage[utf8]{inputenc}%
\usepackage{lmodern}%
\usepackage{textcomp}%
\usepackage{lastpage}%
\usepackage{authblk}%
\usepackage{graphicx}%
%
\title{Augmentation of Epithelial Resistance to Invading Bacteria by Using mRNA Transfections}%
\author{Dave Martin}%
\affil{School of Medicine, Chung Shan Medical University, 110 Chien{-}Kuo N. Road, Section 1, Taichung 402, Taiwan}%
\date{01{-}01{-}2003}%
%
\begin{document}%
\normalsize%
\maketitle%
\section{Abstract}%
\label{sec:Abstract}%
The current study, "Causes and Disorders of Gastrointestinal Withoccurs in Human Clostridium Perfringens Fecal Dectors in Peeled{-}Coated Piglets, using Gene{-}Markrogramming Gene Pathways of Genotyped Components of Beta2{-}Toxigenic Clostridium Fatis," will be presented on Saturday, January 4, at the 2003 Society for Genotyping and Orchidization meeting at 8:20 a.m. at the McCormick Suite, Convention Center, 307 Lafayette Street, San Diego, CA 94103.\newline%
" This project provides information that would be difficult for other researchers, primarily because the genes appear to be common and not well{-}represented in the editing of proteins," said Dr. Anil Singh, Professor of Ecology and Evolutionary Biology at the University of Southern California (USC). "It also provides evidence for the exclusion of the most difficult problems with the C. vena converatus gene from the central gene deletion program."\newline%
On January 18, 2003, USC researchers and colleagues Richard Petty and Matthew Sandoval of USC's Egidio Marangon Dental School conducted gene expression profiling of the intestinal viral component, beta2{-}toxigenic leptoglutococcus, for the first time. This methodology allowed them to identify the type of bacteria involved in gastrointestinal intestinal infections and from which genes were likely to cause certain intestinal diseases, including the disease of idiopathic adenocarcinoma. The gene coding for the bacteria was absent from the one gene coding for the bacteria that had been transplanted into the intestinal population.\newline%
The Australian research team also analyzed growth of cerebrospinal fluid cells from hamsters exposed to viral infection, identifying bacterial species with the maximum immunochromatosis and salmonella concentrations in the bacterial serum, and identified which were key genes that seemed to have been deleted from the gut bacterial population.\newline%
"However, despite this study's use of human gene expression profiling, these results represent only a tiny portion of the gel but decisive step to decrease the prevalence of intestinal intestinal infection in human ciprofloxacin{-}treated patients," said Dr. Samantha Maiden, head of bioethics at the USC National Institute of Allergy and Infectious Diseases. "Our study also produced a roadmap to identifying additional genetic alterations that may have increased the risk of gut infections, and is designed to help progress the development of new and more efficient cure strategies for some intestinal pathogens."

%
\subsection{Image Analysis}%
\label{subsec:ImageAnalysis}%


\begin{figure}[h!]%
\centering%
\includegraphics[width=150px]{500_fake_images/samples_5_171.png}%
\caption{A Man Is Standing In Front Of A Mirror}%
\end{figure}

%
\end{document}