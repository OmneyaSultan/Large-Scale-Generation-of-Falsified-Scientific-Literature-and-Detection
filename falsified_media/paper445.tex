\documentclass{article}%
\usepackage[T1]{fontenc}%
\usepackage[utf8]{inputenc}%
\usepackage{lmodern}%
\usepackage{textcomp}%
\usepackage{lastpage}%
\usepackage{authblk}%
\usepackage{graphicx}%
%
\title{Aucubin, a naturally occurring iridoid glycoside inhibits TNF{-}\_{-}induced inflammatory responses through suppression of NF{-}\_B activation in 3T3{-}L1 adipocytes}%
\author{Eric Mclean}%
\affil{Department of Biochemistry and Molecular Biology, Bengbu Medical College, Bengbu, Anhui, China}%
\date{01{-}01{-}2014}%
%
\begin{document}%
\normalsize%
\maketitle%
\section{Abstract}%
\label{sec:Abstract}%
SENSUSES: Sugar encourages beta cells to produce at least a 24 degree dehydrating reaction. Beta cells run faster and consume more and, over the course of a 24 degree response, glucose drops into their heart cells. To avoid the dramatic rise in blood sugar that occurs during the 24 degree reaction, low{-}glycemic food intake is recommended.\newline%
BEEFING: Exposure to in{-}vitro proteins can increase potassium levels in the rat digestive tract, improve blood flow, reduce cholesterol levels and slightly decrease insulin resistance. Protein levels increase in the rat gene OLYSIGMA (alpha{-}hypothalamic{-}pituitary{-}gamma), which transmits glucose to the oesophagus in the kidney and liver. Oxidative stress makes beta cells less likely to convert glucose into the protein responsible for causing lower hemoglobin levels and hyperglycemia. Small blood pumps in the kidneys signal or correct the pathophysiological response.\newline%
WHY: Blood cell metabolism seems to accelerate when glucose levels drop below a low 24 degree response. A few weeks of hypoglycemia will result in an elevated blood sugar level. Furthermore, hypoglycemia will lead to irreversible metabolic disorders including hemoglobin C{-}reactive protein deficiency and hypoglycemia related kidney inflammation. This scientific finding makes a compelling case for the need for inflammation based treatments for insulin resistance, hyperglycemia and hypoglycemia.

%
\subsection{Image Analysis}%
\label{subsec:ImageAnalysis}%


\begin{figure}[h!]%
\centering%
\includegraphics[width=150px]{500_fake_images/samples_5_50.png}%
\caption{A Cat Is Sitting On The Floor In A Room}%
\end{figure}

%
\end{document}